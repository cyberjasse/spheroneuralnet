\documentclass[a4paper, 12pt]{report}
\usepackage[utf8]{inputenc}
\usepackage[left=2cm,right=2cm,top=2cm,bottom=2cm]{geometry}
\usepackage[pdftex]{graphicx}
\newcommand{\dure}[1]{\item En \textbf{#1},}
\newcommand{\rul}{\rule{5cm}{0.3mm}}

\begin{document}
\newcommand{\spherotitle}[2]{
\begin{titlepage}
\begin{center}
\includegraphics[scale=1.50]{../UMONS.jpg}\\[0.4cm]
\includegraphics[scale=0.30]{../FS_Logo.jpg}\\[3cm]
{\Large Un réseau de neurones pour la}\\
\includegraphics[scale=0.30]{../sphero.jpg}\\
\rule{8cm}{0.5mm}\\[0.5cm]
{\huge \bfseries #1}\\[0.2cm]
\rule{8cm}{0.5mm}\\[7cm]
% Author and supervisor. Come from http://www.jujens.eu/posts/2013/Oct/20/latex-page-garde/
    \begin{minipage}{0.4\textwidth}
      \begin{flushleft} \large
        Jason \textsc{Bury}\\
        130538\\
        Master 1 en\\Sciences informatiques\\
      \end{flushleft}
    \end{minipage}
    \begin{minipage}{0.4\textwidth}
      \begin{flushright} \large
        \emph{Codirecteurs :}~~\\M. Pierre \textsc{Hauweele}\\M. Hadrien \textsc{Mélot}\\
        \emph{Rapporteur : }~~\\M. Tom \textsc{Mens}
      \end{flushright}
    \end{minipage}
   \vfill
  {\large #2}
\end{center}
\end{titlepage}
}

\spherotitle{Cahier de charges}{16 octobre 2016}

\section*{Exigences fonctionnelles}
L'utilisateur sera amené à encoder une trajectoire dans un fichier. C'est-à-dire encoder le chemin à suivre ainsi que les vitesses à atteindre au cours de la trajectoire.
Puis, grâce à un réseau de neurones préalablement entraîné sur un sol donné, le programme devra envoyer fréquemment des commandes à la Sphero2.0\footnote{http://www.sphero.com/sphero} pour qu'elle suive la trajectoire encodée, à moins que l'utilisateur n'ait encodé une trajectoire physiquement impossible pour la Sphero (comme un angle droit à effectuer à 10m/s).

\section*{Ordre des tâches}
Voici un résumé des tâches planifiées, par ordre chronologique:
\begin{enumerate}
 \dure{5 jours} Lecture de documents sur la commande par réseau de neurones.
 \dure{5 jours} Comparaison des différentes API.
 \dure{4 jours} Stream de données de et vers la Sphero.
 \dure{1 jour } Mesure des latences des paquets streamés.
 \dure{3 jours} Outil de visualisation. (Odométrie, trajectoire réelle, vitesses)
 \dure{1 jours} Visualisation d'historique états/commandes.
 \dure{3 jours} Pré-rapport de janvier.\\ \rul\textbf{2\textsuperscript{ème} quadri}\rul
 \dure{2 jour } Suivi de la trajectoire de la Sphero.
 \dure{1 jours} Lecture d'un document sur l'implémentation d'un réseaux de neurones.
 \dure{4 jours} Réalisation d'un réseau de neurones artificiel adéquat.
 \dure{1 jour } Utilisation du réseau de neurones sur un problème jouet.\footnote{Permet de tester le bon fonctionnement du pouvoir d'approximation universelle de fonction du réseau de neurones et de sa paramétrisation.}
 \dure{1 jour } Mesure de la latence des prédictions.
 \dure{1 jour } Rapport sur l'apprentissage en fonction des paramètres du réseau.
 \dure{8 jours} Collecte des données utiles pour l'apprentissage en balayant toutes les configurations possibles.\footnote{Peut-être d'abord restreint sur de faibles vitesses.}
 \dure{2 jours} Chargement et enregistrement d'un fichier contenant les poids et paramètres d'un réseau de neurones.
 \dure{2 jours} Rapport sur le réglage des paramètres du réseau.
 \dure{1 jours} Lecture d'un fichier d'encodage de trajectoire.
 \dure{4 jours} Outil pour tracer la trajectoire.
 \dure{5 jours} Version finale du programme.
 \dure{6 jours} Rapport final.
\end{enumerate}

\end{document}

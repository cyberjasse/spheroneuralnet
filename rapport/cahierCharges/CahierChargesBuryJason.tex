\documentclass[a4paper, 12pt]{report}
\usepackage[utf8]{inputenc}
\usepackage[left=2cm,right=2cm,top=2cm,bottom=2cm]{geometry}
\usepackage[pdftex]{graphicx}

\begin{document}
\newcommand{\spherotitle}[2]{
\begin{titlepage}
\begin{center}
\includegraphics[scale=1.50]{../UMONS.jpg}\\[0.4cm]
\includegraphics[scale=0.30]{../FS_Logo.jpg}\\[3cm]
{\Large Un réseau de neurones pour la}\\
\includegraphics[scale=0.30]{../sphero.jpg}\\
\rule{8cm}{0.5mm}\\[0.5cm]
{\huge \bfseries #1}\\[0.2cm]
\rule{8cm}{0.5mm}\\[7cm]
% Author and supervisor. Come from http://www.jujens.eu/posts/2013/Oct/20/latex-page-garde/
    \begin{minipage}{0.4\textwidth}
      \begin{flushleft} \large
        Jason \textsc{Bury}\\
        130538\\
        Master 1 en\\Sciences informatiques\\
      \end{flushleft}
    \end{minipage}
    \begin{minipage}{0.4\textwidth}
      \begin{flushright} \large
        \emph{Codirecteurs :}~~\\M. Pierre \textsc{Hauweele}\\M. Hadrien \textsc{Mélot}\\
        \emph{Rapporteur : }~~\\M. Tom \textsc{Mens}
      \end{flushright}
    \end{minipage}
   \vfill
  {\large #2}
\end{center}
\end{titlepage}
}

\spherotitle{Cahier de charges}{16 octobre 2016}

\section*{Exigences fonctionnelles}
L'utilisateur sera ammené à encoder une trajectoire dans un fichier. C'est à dire encoder le chemin à suivre ainsi que les vitesses à atteindre au cours de la trajectoire.
Puis, grâce à un réseau de neurones préalablement entraîné sur un sol donné, le programme devra envoyer fréquemment des commandes à la Sphero2.0\footnote{http://www.sphero.com/sphero} pour qu'elle suive la trajectoire encodée, à moins que l'utilisateur n'ait encodé une trajectoire physiquement impossible pour la Sphero (comme un angle droit à effectuer à 10m/s).

\section*{Exigences prioritaires}
Voici les exigeances demandées par ordre décroissant de priorité:
\begin{enumerate}
 \item Stream de données de et vers la Sphero.
 \item Réalisation d'un réseau de neurones artificiel adéquat.
 \item Utilisation du réseau de neurones sur un problème jouet.\footnote{Permet de tester le bon fonctionnement du pouvoir d'approximation universelle de fonction du réseau de neurone et de sa paramètrisation.}
 \item Collecte des données utiles pour l'apprentissage en balayant toute les configurations possibles.
 \item Lecture de fichier contenant les poids et paramètres d'un réseau de neurones.
 \item Lecture d'un fichier d'encodage de trajectoire.
 \item Suivi de la trajectoire.
\end{enumerate}

\section*{Exigences secondaires}
Ici la liste des objectifs secondaires par ordre croissant d'intérêt.
\begin{enumerate}
 \item Approximation du processus par réseau de neurones.\footnote{C'est à dire un réseau de neurones qui prédit la prochaine configuration sur base d'une configuration initiale et la commande à exectuer}
 \item Vérification de la faisabilité d'une trajectoire.
 \item Suivi de chemin le plus rapidement possible.
 \item Permettre à l'utilisateur d'encoder la couleur des leds au fil de la trajectoire.
\end{enumerate}


\end{document}

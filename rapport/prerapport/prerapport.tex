\documentclass[12pt,a4paper,oneside, titlepage]{article}
\usepackage[left=2cm,right=2cm,top=2cm,bottom=2cm]{geometry}
\usepackage[nottoc, notlof, notlot]{tocbibind}
\usepackage[pdftex]{graphicx}
\usepackage[final]{pdfpages}
\usepackage[frenchb]{babel}
\usepackage[utf8]{inputenc}
\usepackage{hyperref}
\usepackage{cite}
\begin{document}
\newcommand{\spherotitle}[2]{
\begin{titlepage}
\begin{center}
\includegraphics[scale=1.50]{../UMONS.jpg}\\[0.4cm]
\includegraphics[scale=0.30]{../FS_Logo.jpg}\\[3cm]
{\Large Un réseau de neurones pour la}\\
\includegraphics[scale=0.30]{../sphero.jpg}\\
\rule{8cm}{0.5mm}\\[0.5cm]
{\huge \bfseries #1}\\[0.2cm]
\rule{8cm}{0.5mm}\\[7cm]
% Author and supervisor. Come from http://www.jujens.eu/posts/2013/Oct/20/latex-page-garde/
    \begin{minipage}{0.4\textwidth}
      \begin{flushleft} \large
        Jason \textsc{Bury}\\
        130538\\
        Master 1 en\\Sciences informatiques\\
      \end{flushleft}
    \end{minipage}
    \begin{minipage}{0.4\textwidth}
      \begin{flushright} \large
        \emph{Codirecteurs :}~~\\M. Pierre \textsc{Hauweele}\\M. Hadrien \textsc{Mélot}\\
        \emph{Rapporteur : }~~\\M. Tom \textsc{Mens}
      \end{flushright}
    \end{minipage}
   \vfill
  {\large #2}
\end{center}
\end{titlepage}
}

\newcommand{\terminologie}{
\begin{figure}
 \begin{center}
 \underline{\hypertarget{terminologie}{Terminologie}}\\
 \begin{tabular}{|l|l|}
  \hline
  Terme & Nom anglais complet\\
  \hline
  ANN & Artificial Neural Network\\
  ELM & Extreme Learning Machine\\
  FNN & Feedforward Neural Networks\\
  MLP & Multi-Layer Perceptron\\
  RBF & Radial Basis Function\\
  CNN & Convolutional neural Network\\
  RNN & Recurrent Neural Network\\
  SRN & Simple Recurrent Network\\
  ESN & Echo State Network\\
  LSM & Liquid State Machine\\
  SNN & Spiking Neural Network\\
  LSTM & Long Short-Term Memory\\
  BRNN & Bi-directional Recurrent Neural Network\\
  RMLP & Recurrent Multilayer Perceptron\\
  \hline
 \end{tabular}
 \end{center}
 \caption{Terminologie des réseaux de neurones dans la littérature anglophone}
 \label{terminologie}
\end{figure}
}

\newcommand{\termi}[1]{\hyperlink{terminologie}{\uppercase{#1}} }
\newcommand{\rbf}{\termi{rbf}}
\newcommand{\mlp}{\termi{mlp}}

\newcommand{\rna}{\hyperlink{rna}{RNA} }
\newcommand{\ubf}[1]{\textbf{\underline{#1}}}
\newcommand{\enum}[1]{``#1''}%{\og\textbf{#1}\fg}
\newcommand{\captionsource}[2]{
  \caption[{#1}]{
    #1
    \\\hspace{\linewidth}
    \textbf{Source:} #2
  }
}

\spherotitle{Pré-rapport}{Décembre 2016}
\tableofcontents
\section{Introduction}
\section{Principaux réseaux de neurones artificiels}
\subsection{Perceptron mutli-couches}
\subsection{Base radiale}
\section{Architectures}
\section{Design du vecteur d'entrée et de sortie}
\section{Choix de l'architecture}
Le réseau le mieux adapté pour notre problème est un réseau de neurones à base radial.
En effet, les \rbf sont moins sensibles au bruit que les \mlp \cite{adversarial,Gauthier}.%gauthier p 39,40
%TODO pourquoi
Un inconvénient des \rbf, celui du nombre de neurones cachés qui croît avec la dimension et la taille du vecteur d'entrée, ne pose pas de problème dans notre cas
car le nombre de dimensions est faible
et le domaine d'entrée est assez faible pour chaque dimensions.
%TODO montrer avant les dimensions et les domaines
\section{API disponibles}
\section{Choix de l'API}
\section{Conclusion}
\subsection*{Remerciements}
\bibliographystyle{unsrt-fr}
\bibliography{../bibli}
\end{document}

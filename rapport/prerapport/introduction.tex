\section{Introduction}
\subsection{Énoncé}
Ce projet consiste à implémenter un \emph{réseau de neurones artificiels} \hypertarget{rna}{(RNA)} pour commander la Sphero.
Il devra effectuer n'importe quelle trajectoire qui peut être à grande vitesse et aussi gérer les dérapages.
La Sphero sera commandée sur un sol plat sans obstacle.
\subsection{Présentation de la Sphero}
La Sphero est une boule robotisée téléguidée, connectée par Bluetooth.
À l'intérieur, il y a deux moteurs: un pour faire tourner le poid et l'autre pour changer l'orientation de l'appareil.
En faisant tourner le poid, la Sphero déplace son centre de gravité en dehors de sa base de sustentation, faisant ainsi rouler l'appareil.
\subsection{Avantages d'un réseau de neurones}
\begin{itemize}
 \item \textbf{parallélisme}: L'apprentissage et la génération de vecteur de sortie est massivement parallélisable dans un RNA.\cite{corelet,Haykin}
 \item \textbf{Généralisation}: Répond de manière raisonnable à une entrée non rencontrée durant la phase d'apprentissage.\cite{statistica,Haykin}
 \item \textbf{Approximation non-linéaire}: Les RNA présentés dans ce rapport sont des approximateurs universels de fonctions non-linéaires.\cite{Haykin}
 \item \textbf{Adaptabilité}: Un RNA avec apprentissage on-line s'adapte aux changements dans le système.\cite{Haykin}
 \item \textbf{Boite noire}: Un RNA agit comme une boite noire. L'utilisateur n'a pas besoin de connaitre le fonctionnement du réseaux.
 \item \textbf{Tolérance au bruit}: Le bruitage dans la phase d'apprentissage impacte peu les performances d'un RNA.\cite{Haykin}
\end{itemize}
C'est grâce à ces avantages qu'un réseaux de neurones artificiels peut être intéressant pour commander la Sphero.
En effet, Il est très difficile de générer les commandes de manières analytique fidèle à la réalité à cause des trop nombreux paramètres à gèrer (Centre de gravité changeant, frottement, dérapages, équilibre, défauts,...).
Grâce au fait qu'un RNA agit comme une boite noire et est approximateur universel de fonction, nous pouvons nous passer de la conception d'un simulateur ou d'une tentative de formule analytique pour génerer les commandes.
De plus, si l'apprentissage se fait on-line, le réseau de neurones pourra s'adapter si le coefficient de frottement du sol change ou si il y une modification à la Sphero.
Et enfin, nous aurons invévitablement des bruitages dans les données fournies par les capteurs.
\subsection{Design du vecteur d'entrée}
\subsubsection{Données utiles}
Avant de choisir le réseau de neurones à utiliser pour générer des commandes pour la Sphero, il est important d'identifier ce que nous souhaitons obtenir en sortie et les informations que nous pouvons fournir en entrée en omettant les données inutiles.
Si le vecteur d'entrée est de dimension très grande, alors un réseau \rbf devra avoir un trop grand nombre de neurones cachés\cite{Gauthier}.
Si nous supposons que la sortie produite peut dépendre non seulement de l'input actuel mais aussi des entrées précédentes, alors un réseau récurrent devra être utilisé.

La Sphero peut nous fournir\cite{SDKofficiels}:
\begin{itemize}
 \item Sa position grâce à un odomètre (cm),
 \item Son vecteur vitesse (mm/s),
 \item Son vecteur d'accélération grâce à un accéleromètre (mG),
 \item Son orientation (-179$\rightarrow$180 degrés),
 \item Son niveau de charge (Pas de poucentage mais label \enum{Chargement}, \enum{OK}, \enum{Bas} et \enum{Critique}),
 \item État des LEDs, évènement de collision, voltage de la batterie, nombre de charges, version ...
\end{itemize}

Parmis ces données, nous n'avons pas besoin de l'état des LEDs, version et nombre de charges car ils n'influencent pas la conduite de la Shero.
Dans ce projet, nous ne prenons pas en compte les éventuels collisions.

Le niveau de charge pourrait influencer la conduite.
Nous ne savons pas si la puissance maximale des moteurs diminuent lorsque la charge est trop faible.
Si cela s'avère être le cas, pour ne pas devoir effectuer un entrainement pour chaque niveau de charge différent, on entrainera la Sphero que pour les cas où le niveau de charge est à \enum{OK} et on supposera donc que la conduite sera effectuée qu'avec ce niveau de charge.

Un odomètre n'est pas approprié pour la mesure de position dans notre application.
En effet, la distance parcourue n'est mesurée qu'à partir du roulement du moteur permettant le déplacement.
La distance parcourue par dérapage ne sera pas comptabilisée.
Cela a été vérifié lors de l'expérience menée dans la section \ref{sec:choixdesign}.
Une camera est plus appropriée pour tracer la position réelle de la Sphero.

Pour l'accélération, les commandes à appliquer à l'instant $t$ ne dépendent pas de l'accélération à l'instant $t$ juste avant l'application des commandes.
Mais par contre, cette donnée peut être utile pour anticiper la vitesse de la Sphero au moment où elle reçoit les commandes car, dû à la latence d'envoi de paquet, la vitesse à l'instant $t$ n'est plus la même que celle communiquée.
Si cette donnée impacte peu dans les résultats obtenus, alors nous pourrions décider de nous passer de ce paramètre.
Ajouter cette dimension dans le vecteur d'entrée demandera un plus grand nombre d'exemples pour la phases d'apprentissage car il faut balayer toutes les configurations possibles de vecteur d'entrée.

\newcommand{\inchist}[1]{
 \begin{minipage}{0.48\textwidth}
  \includegraphics[height=8cm]{../figures/perf#1.png}
 \end{minipage}
}
\newcommand{\incbefore}[1]{
 \begin{minipage}{0.48\textwidth}
  \includegraphics[width=8cm]{../figures/perf#1Before.png}
 \end{minipage}
}
\subsubsection{Choix du design}\label{sec:choixdesign}
 %TODO montrer les design repris chez gauthier
On aligne l'axe des abscisses avec l'axe de l'orientation du moteur afin de reduire le domaine d'entrée et donc le nombre d'exemples à créer pour l'apprentissage.
Le vecteur d'entrée contiendra donc le vecteur de la vitesse actuelle, le vecteur de la vitesse à atteindre et la position relative à atteindre dans $T$ secondes où $T$ sera la période à laquelle on échantillonera les données des capteurs.
Nous avons donc un vecteur d'entrée à 6 dimensions (vitesse actuelle en x et y, position relative cible en x et y, acccélération actuelle en x et y).
Nous pourrions réduire d'avantage le domaine d'entrée si nous considérons que la direction et la vitesse à prendre pour atteindre une position est en symétrie orthogonale d'axe x.

Deux possibilités s'offrent à nous pour le vecteur de sortie:\\
\begin{tabular}{|l|l|l|}
 \hline
 \textbf{Vecteur} & \textbf{Avantages} & \textbf{Inconvénients}\\
 \hline
 \{puissance du moteur A & \tabitem Plus précis & \tabitem Stabilisateur désactivé\\
 ~puissance du moteur B\} & \tabitem Permet plus de dérapage & \tabitem Risque de commande\\
  & \tabitem Grande vitesse dans les & \enum{inutiles}\\
  & virages & \tabitem Risque une déstabilisation\\
  & & totale\\
 \hline
 \{vitesse, orientation\} & \tabitem Stabilisateur activé & \tabitem Moins précis\\
  & \tabitem Toutes les commandes ont & \tabitem Évite les dérapages\\
  & un effet &\\
 \hline
\end{tabular}\\
Par commande \enum{inutiles}, nous entendons par là des commandes qui peuvent sembler absurdes.
Par exemple, si la puissance donnée au moteur qui soulève le contre-poids est très élevée alors que la Sphero a une faible vitesse, le contre-poids tournera et se retrouvera dans le sens opposé à celui qui permettait de faire avancer la Sphero.
Et la déstabilisation totale est une perte du repère qui sert à capturer l'orientation actuelle de la Sphero.
En effet, l'orientation de la Sphero est donnée par rapport à un repère, 0 degré, qui est sensé resté constant.
Si le stabilisateur est désactivé, la coque elle-même risque de changer d'orientation.
Modifiant donc le repère.
Ce qui biaisera toutes les orientations capturées par la suite.
C'est pour cela que nous choisirons comme vecteur de sortie la vitesse et l'orientation.

Estimons la valeur optimale que nous pouvons donner à $T$.
Une commande de streaming de données peut être envoyée à la Sphero.
Grâce à cette commande, nous pouvons demander à la Sphero d'envoyer uniquement les données qui nous interessent à une fréquence donnée.
Trois paramètres de cette commande peuvent influencer les performances d'échantillonage:
\begin{enumerate}
 \item Un diviseur (entier) de la fréquence maximale d'échantillonage.
 \item Le nombre d'échantillonage à garder en mémoire avant envoi.
 \item Le nombre de capteur à échantilloner.
\end{enumerate}
Un échantillonage consiste à obtenir seulement les données qui nous intéressent à un instant précis.
La fréquence maximale d'échantillonage, un échantillon à la fois, est de 400Hz.\cite{SDKofficiels}
La vitesse maximale de la Sphero est de 4,5 miles par heure.\cite{product} Ce qui fait environ 2 mètres par seconde.
\begin{figure}
 \centering
 \inchist{5}
 \inchist{10}
 \inchist{20}
 \inchist{40}
 \inchist{80}
 \inchist{100}
 \caption{Histogrammes de temps de réception de packet streamé. Générés par script R}
 \label{histogrammes}
\end{figure}

À la Figure \ref{histogrammes} sont repris les histogrammes de temps de réceptions de 500 paquets pour différentes fréquences d'échantillonage.
Les paquets invalides, donc de checksum incorrect ou de taille trop courte, sont rejetés.
Ce qui explique les pics sur $2T$.
En plus du streaming, pour chaque paquet reçu, un paquet est envoyé à la Sphero pour changer les leds.
Cela nous permet d'obtenir des résultats qui se rapprocheront de ce que nous obtiendrons si nous envoyons une nouvelle commande à chaque période $T$.
Les fréquences de streaming choisis sont des diviseurs de 400 car le démarage d'un streaming demande en paramètre un diviseur de 400.\ref{SDKofficiels}
La Sphero divise ensuite 400 par ce paramètre pour obtenir la fréquence demandée.
La configuration de l'ordinateur utilisé est la suivante:
\begin{itemize}
 \item AMD Athlon(tm) X2 Dual-Core QL-64
 \item Carte mère EI Capitan
 \item 3 Go de RAM DDR2
 \item disque dur ST9250827AS 5400rpm
 \item Adapteur USB bluetooth BT009x, transfert 1Mbs.
 \item OS: Linux 4.4.0-57-generic, Ubuntu 16.04.4
\end{itemize}

Pour les critères de choix d'une valeur pour $T$, deux critères sont pris en compte:
\begin{enumerate}
 \item La stabilité du streaming. Moins la densité de probabilité de temps de réception de paquet est concentrée sur le voisinage de $T$, moins la solution sera précise.
 \item Le temps de réaction. Plus $T$ est grand, moins la trajectoire de la Sphero sera fidèle à la trajectoire voulue.
 Nous pouvons compenser ce manque de précision par un vecteur d'entrée plus complexe mais la solution converge alors moins vite et un ensemble d'apprentissage plus conséquent doit être fourni.
\end{enumerate}

\begin{figure}
 \centering
 \incbefore{5}
 \incbefore{10}
 \incbefore{20}
 \incbefore{40}
 \incbefore{80}
 \caption{Dates de réception de paquet - dates esperées. Générés par script R}
 \label{befores}
\end{figure}
Une autre façon d'observer la stabilité d'un streaming est d'enregistrer la date de réception de chaque paquet, de leur soustraire la date esperée de réception et puis de reporter toutes ces valeurs dans un graphique.
Toutes les parties du graphique représentant une droite horizontale représentent un moment pendant laquelle la période de réception de paquets valides est stable.
Si on observe bond vertical vers le haut de $T$ secondes, alors il un paquet a été rejeté à ce moment là.
Ce graphiques sont reportés dans la Figure \ref{befores}.

On observe que la fréquence de rejet de paquet est anormalement régulière.
On observe presque 10 paquets valides entre paquet rejeté.
tous ces paquets ont été rejeté car leur taille était insuffisante.
Pour empêcher que ce manque de précision nuit à la convergence du réseau de neurones, nous pourrions, dans l'ensemble d'apprentissage, retirer les instances dont le temps mis pour atteindre l'état suivant est de environ $2T$.
En rejetant donc toutes ces données, nous pouvons donc ne plus prendre en compte du pic sur $2T$ dans les histogrammes de la Figure \ref{histogrammes}.
On observe dès lors que le plus stable celui à 80Hz car le pic en $T$ est plus haut que celui des autres et avec une largeur presque pareille.
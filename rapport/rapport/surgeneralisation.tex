\subsection{Le problème du surajustement}
\subsubsection{Le phénomène}
Lorsqu!un réseau de neurones à apprentissage supervisé apprend de trop nombreuses fois à partir du même set d'exemple, alors il peut se produire un phénomène appelé \emph{surajustement}\cite{statistica}.
Le réseau devient dès lors de moins en moins efficace sur des entrées non rencontrées pendant les phases d'apprentissage.
La Figure \ref{interruption} montre qu'au fil des cycles d'apprentissage, l'erreur diminue.
Mais à un certain moment, l'erreur sur des paires entrée/sortie non fournies pendant la phase d'apprentissage augmente. 
C'est à partir de ce moment là que le réseau de neurones surajuste.
\begin{figure}
 \centering
 \includegraphics[scale=0.6]{../figures/surgeneralisation.jpg}
 \caption{Interruption prématurée. \textbf{Source}: STATISTICA Réseaux de Neurones Automatisés (SANN)\cite{statistica}}
 \label{interruption}
\end{figure}
\subsubsection{Les solutions}
\ssstitle{Intérruption prématurée}
Tout d'abord séparons notre ensemble de paires entrée/sortie en deux ensembles.
L'un est nommé \emph{ensemble d'apprentissage} et l'autre \emph{ensemble de test}.
Chacun de ces ensembles doit balayer toutes les valeurs d'entrées possibles.
Ensuite à chaque cycle, on va d'abord effectuer un apprentissage pour chaque paires de l'ensemble d'apprentissage.
Puis on génère des sorties à partir des entrées de l'ensemble de test.
On compare l'erreur de test à celui de l'étape précédente.
L'erreur de test est l'erreur calculée sur les sorties générées et les sorties attendues de l'ensemble de test.
Si l'erreur de test est plus grand que celui du cycle précédent, alors on arrête les cycles d'apprentissage.
Dans la Figure \ref{interruption}, on observe que l'interruption prématurée se produit au moment où son \emph{pouvoir de généralisation}, c'est à dire son efficacité pour des entrées jamais rencontrées, est le plus élevé.\\

Dans la réalité, la courbe d'erreur de test n'est pas aussi lisse que dans la Figure \ref{interruption} mais présente un certain bruit.
Il faut donc pouvoir s'assurer que l'interruption prématurée a bien lieu sur le minimum global de l'erreur de test.
Comparer l'erreur de test à celui du cycle précédent permet de détecter seulement un minimum local.
\ssstitle{Modération des poids}
Cette méthode consiste à pénaliser l'utilisation de trop grand poids.
Pour ce faire, on modifie le calcul d'erreur pour y prendre en compte leur valeur\cite{statistica}.
Soit $E$ l'erreur, soit $W$ un vecteur contenant tous les poids (ne pas prendre les biais en compte), la nouvelle valeur de l'erreur est \[E_{new} = E + \frac{\sigma}{2}W^{T}W\]
où $\sigma$ est une constante appelée \emph{constante de modération}.
Un $\sigma$ trop petit ne permet pas d'éviter le surajustement et un $\sigma$ trop grand empêche la généralisation.

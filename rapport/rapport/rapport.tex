\documentclass[12pt,a4paper,oneside, titlepage]{article}
\usepackage[left=2cm,right=2cm,top=2cm,bottom=2cm]{geometry}
\usepackage[pdftex]{graphicx}
\usepackage[final]{pdfpages}
\usepackage[frenchb]{babel}
\usepackage[utf8]{inputenc}
\usepackage{cite}
\usepackage{hyperref}
\usepackage{amsmath}
\usepackage{amsthm}
\begin{document}
\newcommand{\spherotitle}[2]{
\begin{titlepage}
\begin{center}
\includegraphics[scale=1.50]{../UMONS.jpg}\\[0.4cm]
\includegraphics[scale=0.30]{../FS_Logo.jpg}\\[3cm]
{\Large Un réseau de neurones pour la}\\
\includegraphics[scale=0.30]{../sphero.jpg}\\
\rule{8cm}{0.5mm}\\[0.5cm]
{\huge \bfseries #1}\\[0.2cm]
\rule{8cm}{0.5mm}\\[7cm]
% Author and supervisor. Come from http://www.jujens.eu/posts/2013/Oct/20/latex-page-garde/
    \begin{minipage}{0.4\textwidth}
      \begin{flushleft} \large
        Jason \textsc{Bury}\\
        130538\\
        Master 1 en\\Sciences informatiques\\
      \end{flushleft}
    \end{minipage}
    \begin{minipage}{0.4\textwidth}
      \begin{flushright} \large
        \emph{Codirecteurs :}~~\\M. Pierre \textsc{Hauweele}\\M. Hadrien \textsc{Mélot}\\
        \emph{Rapporteur : }~~\\M. Tom \textsc{Mens}
      \end{flushright}
    \end{minipage}
   \vfill
  {\large #2}
\end{center}
\end{titlepage}
}

\newcommand{\terminologie}{
\begin{figure}
 \begin{center}
 \underline{\hypertarget{terminologie}{Terminologie}}\\
 \begin{tabular}{|l|l|}
  \hline
  Terme & Nom anglais complet\\
  \hline
  ANN & Artificial Neural Network\\
  ELM & Extreme Learning Machine\\
  FNN & Feedforward Neural Networks\\
  MLP & Multi-Layer Perceptron\\
  RBF & Radial Basis Function\\
  CNN & Convolutional neural Network\\
  RNN & Recurrent Neural Network\\
  SRN & Simple Recurrent Network\\
  ESN & Echo State Network\\
  LSM & Liquid State Machine\\
  SNN & Spiking Neural Network\\
  LSTM & Long Short-Term Memory\\
  BRNN & Bi-directional Recurrent Neural Network\\
  RMLP & Recurrent Multilayer Perceptron\\
  \hline
 \end{tabular}
 \end{center}
 \caption{Terminologie des réseaux de neurones dans la littérature anglophone}
 \label{terminologie}
\end{figure}
}

\newcommand{\termi}[1]{\hyperlink{terminologie}{\uppercase{#1}} }
\newcommand{\rbf}{\termi{rbf}}
\newcommand{\mlp}{\termi{mlp}}

\newcommand{\rna}{\hyperlink{rna}{RNA} }
\newcommand{\ubf}[1]{\textbf{\underline{#1}}}
\newcommand{\enum}[1]{``#1''}%{\og\textbf{#1}\fg}
\newcommand{\captionsource}[2]{
  \caption[{#1}]{
    #1
    \\\hspace{\linewidth}
    \textbf{Source:} #2
  }
}

\spherotitle{Rapport}{Année 2016-2017}
\tableofcontents
\newpage
\section{Introduction}
\subsection{Énoncé}
Ce projet consiste à implémenter un \emph{réseau de neurones artificiels} \hypertarget{rna}{(RNA)} de commander la Sphero.
Il devra effecter n'importe quel trajectoire et aussi des trajectoire à grande vitesse et gèrer les dérapages.
La Sphero sera commandé sur un sol plat sans obstacle.
\subsection{Avantages d'un réseau de neurones}
\begin{itemize}
 \item \textbf{parallélisme}: L'apprentissage et la génération de vecteur de sortie est massivement parallélisable dans un RNA.\cite{corelet,Haykin}
 \item \textbf{Généralisation}: Répond de manière raisonnable à une entrée non rencontrée durant la phase d'apprentissage.\cite{statistica,Haykin}
 \item \textbf{Approximation non-linéaire}: Les RNA présentés dans ce rapport sont des approximateurs universels de fonctions non-linéaires.\cite{Haykin}
 \item \textbf{Adaptabilité}: Un RNA avec apprentissage on-line s'adapte aux changements dans le système.\cite{Haykin}
 \item \textbf{Boite noire}: Un RNA agit comme une boite noire. L'utilisateur n'a pas besoin de connaitre le fonctionnement du réseaux.
 \item \textbf{Tolérance au bruit}: Le bruitage dans la phase d'apprentissage impacte peu les performances d'un RNA.\cite{Haykin}
\end{itemize}
C'est grâce à ces avantages qu'un réseaux de neurones artificiels peut être intéressant pour commander la Sphero.
En effet, Il est très difficile de générer les commandes de manières analytique fidèle à la réalité à cause des trop nombreux paramètres à gèrer (Centre de gravité changeant, frottement, dérapages, équilibre, défauts,...).
Grâce au fait qu'un RNA agit comme une boite noire et est approximateur universel de fonction, nous pouvons nous passer de la conception d'un simulateur ou d'une tentative de formule analytique pour génerer les commandes.
De plus, si l'apprentissage se fait on-line, le réseau de neurones pourra s'adapter si le coefficient de frottement du sol change ou si il y une modification à la Sphero.
Et enfin, nous aurons invévitablement des bruitages dans les données fournies par les capteurs.

\section{Modèles de réseaux de neurones artificiels}
\terminologie
\subsection{Perceptron mutli-couches}
Dans la suite, nous désignerons un \emph{perceptron multi-couches} par sa terminologie anglaise: \mlp pour \enum{MultiLayer Perceptron}.
\subsubsection{Neurone}
La Figure \ref{neuronemlp} schématise le travail d'un neurone d'indice $k$.
\begin{figure}
 \centering
 \includegraphics[scale=0.5]{../figures/neurone.jpg}
 \caption{Un neurone artificiel. \textbf{Source}: Haykin\cite{Haykin}}
 \label{neuronemlp}
\end{figure}
Un neurone effectue tout d'abord une somme pondérée de ses entrées \[v_k = b_k+\sum_{j=1}^{m}x_{j}w_{kj}\] où $x$ est le vecteur d'entrée de dimensions $m$.\\
Chaque terme $x_j$ est mutiplié par un poids $w_{kj}$.
Ce sont les poids qui seront modifiés lors de la phase d'apprentissage.
Le biais $b_k$ est souvent ajouté à la pondération.
Mais pour simplifier les formules, nous pouvons considèrer $b_k$ comme étant l'entrée $x_0 = 1$ de poid fixe $w_{k0} = 1$.
Et la somme pondérée est donc maintenant de la forme \[v_k = \sum_{j=0}^{m}x_{j}w_{kj}\]
Ensuite le neurone applique la \emph{fonction d'activation} $\phi$ sur la somme pondérée.
Le domaine de $y$ est généralement $[0,1]$ ou $[-1,1]$.\cite{Haykin,statistica}
La fonction $\phi$ utilisée dépend du problème qu'on veut résoudre (Table \ref{mlpfonc}).
Par exemple, la fonction Softmax est utilisée en classification.\cite{statistica}%TODO autre fonction

\begin{table}
 \centering
 % tableau de statistica
 \textbf{Fonctions d'activations.} (fonction de $x$)\\
 \begin{tabular}{|l|c|c|}
  Nom & Formule & Image\\
  \hline
  Identité & $x$ & $]-\infty,\infty[$\\
  \hline
  Sigmoïde & $\frac{1}{1+\exp^{-x}}$ & $]0,1]$\\
  \hline
  Tangente hyperbolique & $\frac{\exp^{x}-\exp^{-x}}{\exp^{x}+\exp^{-x}}$ & $]-1,1[$\\
  \hline
  Exponentielle & $e^{-x}$ & $]0,\infty[$\\
  \hline
  Sinus & $\sin{x}$ & $[-1,1]$\\
  \hline
  Softmax & $\frac{\exp^{x}}{\sum{\exp^{x_i}}}$ & $]0,1[$\\
  \hline
 \end{tabular}
 \caption{Fonctions d'activation principalement utilisés dans un MLP. \textbf{Source}: STATISTICA Réseaux de Neurones Automatisés (SANN)\cite{statistica}}
 \label{mlpfonc}
\end{table}
\subsubsection{Structure}
\begin{figure}
 \centering
 \includegraphics[scale=0.5]{../figures/nnstruct.png}
 \caption{Structure MLP à une couche cachée. \textbf{Source}: McCormick\cite{RBFtuto}}
 \label{structuremlp}
\end{figure}
Un \mlp est composé de plusieurs couches (Figure \ref{structuremlp}):
\begin{center}
 couche d'entrée $\rightarrow$ couches cachées $\rightarrow$ couche de sortie.
\end{center}
Un neurone envoie sa sortie vers tous les neurones de la couche suivante.
Il y a autant de neurones d'entrées que la dimension du vecteur d'entrée.
Le $k$\textsuperscript{ième} neurone d'entrée renvoie juste le $k$\textsuperscript{ième} élément du vecteur d'entrée.
Chaque neurone de sortie correspond à une dimension du vecteur de sortie.
Les neurones cachés et neurones de sortie correspondent à la Figure \ref{neuronemlp}.
\subsubsection{Apprentissage supervisé}\label{sec:appmlp}
Il existe plusieurs algorithmes pour changer les poids sur un réseau. Mais le plus connu est l'algorithme de de rétropropagation.\cite{Haykin,Gauthier}
Il s'agit d'un algorithme d'\emph{apprentissage supervisé}.
\begin{definition}
L'apprentissage supervisé est une méthode visant à améliorer un approximateur grâce à un calcul d'erreur (appelé aussi mesure de performance\cite{Gauthier}) comparant une sortie générée avec la sortie attendue.
\end{definition}
Nous avons donc besoin d'un ensemble de paires d'entrées/sorties.
L'algorithme applique la formule développée ci-dessous à tous les neurones sauf les neurones d'entrée.\\

Soit $\phi_i$, la fonction d'activation du neurone $i$. L'erreur entre la sortie produite et la sortie attendue est donnée par $Q$, l'\emph{erreur quadratique} utilisé comme mesure de performance.
\begin{equation} \label{eq:Q}
 Q = \frac{1}{2}\sum_{i}(\phi_i-s_i)^2
\end{equation}
où $i$ est l'indice d'un neurone de sortie et $s_i$ la sortie attendue pour le neurone $i$.

Nous voulons modifier la valeur des poids pour que la prochaine fois que nous prenons la même entrée, l'erreur $Q$ soit plus petite.
Pour chaque poids $W_{ij}$, nous allons calculer le gradient $\partiel{Q}{W_{ij}}$ qui, par définition, indique la façon dont $Q$ varie si $W_{ij}$ augmente d'une valeur $\delta W_{ij}$ infiniment petite.
Ensuite nous \enum{ferons un pas} dans le sens opposé et proportionel au gradient en ajoutant $-\eta\partiel{Q}{W_{ij}}$ à $W_{ij}$ où $\eta$ est une constante appelé \emph{pas du gradient}.
Donc la modification du poid $W_{ij}$ que nous voulons appliquer vaut
\begin{equation}%\label{eq:Delta}
 \Delta W_{ij} = -\eta\partiel{Q}{W_{ij}}
\end{equation}
Déterminons à présent la valeur de ce gradient $\partiel{Q}{W_{ij}}$.
Nous savons que $\phi_i$ est une équation à une variable.
Mais en entrée de cette variable est donnée la somme pondérée $v_i$ et $W_{ij}$ est un des poids de $v_i$.
Par le théorème de dérivation des fonctions composées,
\begin{equation}%\label{eq:gradient}
 \partiel{Q}{W_{ij}} = \partiel{Q}{\phi_i} \partiel{\phi_i}{v_i} \partiel{v_i}{W_{ij}}
\end{equation}
\begin{thm}[dérivation des fonctions composées]
Soit $f:A~\rightarrow~B : y~\rightarrow~f(y)$ et $g:B~\rightarrow~C : x~\rightarrow~g(x)$. Alors la dérivée de $f~\circ~g$ en $x$ vaut
\[\partiel{f}{x} = \partiel{f}{g} \partiel{g}{x}\]
\end{thm}
Posons
\begin{equation}%\label{eq:deltai}
 \delta_i = \partiel{Q}{\phi_i} \partiel{\phi_i}{v_i}
\end{equation}
$\delta_i$ est appelé \emph{contribution à l'erreur} du neurone $i$.
Nous verrons plus tard qu'elle sera utile pour la rétropropagation de la couche prédédente.
Le dévellopement de la formule de rétropropagation a été fait en annexe \ref{sec:eqmlp}.
Et voici ce que nous obtenons au final:\\
\begin{equation}\label{eq:mlpretro}
 \Delta W_{ij} = -\eta \delta_i x_i \text{~où~}\left\{
  \begin{array}{lll}
   \delta_i & = (\phi_i - s_i)\phi'(v_i) & \text{Si~} i \text{~est un neurone de sortie}\\
   \delta_i & = \phi_i'(v_i) \sum_{k=1}^{n} \delta_k W_{ki} & \text{Si~} i \text{~est un neurone caché}
  \end{array}
 \right.
\end{equation}

\subsubsection{Apprentissage non supervisé}
Il existe également des algorithmes d'apprentissage non supervisé.
Ces algorithmes ne cherchent pas à minimiser une erreur mais maximisent un score calculé à partir de la sortie du réseau.
\subsubsection{Apprentissage sur plusieurs MLP}
Lorque l'une des dimensions du vecteur d'entrée est discrète et finie, il est conseillé d'utiliser un \mlp différent par valeur différente sur cette dimension.\cite{Gauthier}
Cette technique peut aussi être utilisée si nous pouvons classer tous les états possibles selon le contexte.
Par exemple, pour la Sphero nous pourrions utiliser un \mlp pour le contexte \enum{en train de déraper} et un autre \mlp pour le contexte \enum{ne dérape pas}.
\subsubsection{Applications}
Les \mlp sont aussi utilisés en commande de robot par caméra.\cite{Pomerleau}
Un \mlp peut aussi être envisagé dans notre application.
Nous lui fournissons une entrée composée d'informations sur l'état actuel du monde, l'état cible et ce réseau de neurones devra nous fournir en sortie la commande à appliquer pour atteindre l'état-cible.

\subsubsection{Fonctions à base radiale}
Nous désignerons un réseau de neurones \emph{fonctions à base radiale} par sa terminologie anglaise: \rbf pour \enum{Rasial Basis Function}.
\newcommand{\factnorm}{\sum_{r=1}^{m}x_{r}}
\ssstitle{Neurone}
Un neurone \textbf{caché} d'un \rbf n'effectue pas de somme pondérée de ses entrées.
Il applique directement sa fonction d'activation $\phi$, une gaussienne de dimension $n$, de moyenne $\mu$ (appelé aussi prototype) et d'écart-type $\sigma$.
\[\phi(x) = e^{-\frac{1}{2}\sum_{k=1}^{n}\frac{(x_k-\mu_{k})^2}{\sigma_{k}^{2}}} \]%\label{eq:cachephi}
%$\phi(x)$ peut se résumer en: \[\phi(x) = e^{-\beta||x-\mu||^2}\]Où $\beta$ est donc un coefficient qui règle la largeur de la courbe en cloche.\\

Un neurone \textbf{de sortie} d'un \rbf n'effectue pas non plus de somme pondérée de ses entrées. Sa fonction d'activation est:
\[\phi(x) = \frac{\sum_{j=1}^{m}W_{j}x_{j}}{\sum_{j=1}^{m}x_{j}}\]%\label{eq:sortiephi}
où $m$ est le nombre de dimension de $x$.
\begin{figure}
 \centering
 \includegraphics[scale=0.7]{../figures/RBFactivation.png}%TODO generer ça soit-même
 \caption{Activation d'un neurone RBF avec différentes valeurs d'écart-type. $\beta=\frac{1}{2\sigma^2}$ \textbf{Source}: McCormick\cite{RBFtuto}}
 \label{rbfactivation}
\end{figure}
La Figure \ref{rbfactivation} représente la sortie d'un neurone \rbf où $\mu$ et $x$ sont de dimension 1 et $\mu = 0$.\\
Pour résumer, un neurone \rbf renvoie une valeur indiquant la similarité entre l'entrée et son prototype.
\ssstitle{Structure}
\begin{figure}
 \centering
 \includegraphics[scale=0.5]{../figures/RBFstruct.png}
 \caption{Structure RBF. \textbf{Source}: McCormick\cite{RBFtuto}}
 \label{structurerbf}
\end{figure}
La structure d'un \rbf est comme celle du \mlp sauf qu'il n'y a qu'une seule couche cachée (Figure \ref{structurerbf}).
\ssstitle{Apprentissage}
Dans ce réseau, les paramètres qui seront modifiés lors de l'apprentissage sont les poids des neurones de sortie et les moyennes et écart-types des neurones cachés.
L'algorithme de rétropropagation peut être utilisé pour un apprentissage supervisé d'un \rbf.
Reprenons \eqref{eq:Q}, l'erreur quadratique $Q$ défini dans la section \ref{sec:appmlp}.
En reprenant le même raisonnement que la rétropropagation dans un \mlp,
on va faire un pas de $\eta$ dans le sens opposé et proportionnel au gradient pour chaque poids $W_{j}$ des neurones de sortie mais aussi des prototypes $\mu_j$ et écart-types $\sigma_j$ des neurones cachés.
C'est à dire que les paramètres seront modifiés de la sorte pour le neurone $i$:
\[\Delta W_{ij} = -\eta \partiel{Q}{W_{ij}}\]
\[\Delta \mu_{ij} = -\eta \partiel{Q}{\mu_{ij}}\]
\[\Delta \sigma_{ij} = -\eta \partiel{Q}{\sigma_{ij}}\]

Le dévellopement des formules de rétropropagation a été fait en annexe \ref{sec:eqrbf}.
Et voici ce que nous obtenons au final:\\
Pour un neurone de sortie :
\[\Delta W_{ij} = -\eta (\phi_i - s_i) \frac{x_j}{\factnorm}\]
Pour un neurone caché, posons $x_r$ l'entrée de $j$ provenant de $i$ (c'est à dire $\phi_i = x_r$),\\
posons $n$ la dimension de l'entrée de $j$,\\
posons aussi $R = \sum_{k=1}^{n}x_k$.
\[\Delta\mu_{ik} = -\eta \left[\sum_{i}(\phi_j - s_j) \frac{1}{R^2} \left(W_{jr}R - \sum_{k=1}^{n}W_{jk}x_k\right)\right] \phi_i\frac{x_k-\mu_{ik}}{\sigma_{ik}^2}\]
\[\Delta \sigma_{ik} = -\eta \left[\sum_{i}(\phi_j - s_j) \frac{1}{R^2} \left(W_{jr}R - \sum_{k=1}^{n}W_{jk}x_k\right)\right] \phi_i \frac{(x_k-\mu_{ik})^2}{\sigma_{ik}^3}\]

\ssstitle{Applications}
Un \rbf peut aussi servir pour la commande de robot\cite{Gauthier}.
Un des inconvénients des \rbf est que le nombre de neurones cachés croît avec la dimension et la taille du vecteur d'entrée puisqu'un neurone caché s'active seulement pour des entrées dans le voisinage de son prototype.
Nous pouvons raisonablement envisager ce modèle pour notre application puisque la description de l'état actuel et de l'état cible contiendra typiquement une coordonnée, un vecteur vitesse, une orientation et éventuellement l'accéleration, etc...
La taille de l'entrée sera donc relativement petite. Ce sera lors de la phase pratique qu'on évaluera le nombre de neurones cachés nécessaires.
De plus, les \rbf sont moins sensibles au bruit\cite{adversarial}.
%\input{cnn}
\subsection{Récurrent (RNN)}
\subsubsection*{Structure}
\begin{figure}
 \centering
 \includegraphics[scale=0.5]{../figures/structurermlp.jpg}
 \caption{Structure RLMP. \textbf{Source}: Haykin. p795\cite{Haykin}}
 \label{structurermlp}
\end{figure}
\begin{figure}
 \centering
 \includegraphics[scale=0.5]{../figures/neuronermlp.jpg}
 \caption{Neurone caché d'un RMLP}
 \label{neuronermlp}
\end{figure}
Un réseau de neurones récurrent est un réseaux présentant des boucles dans sa structure.
Des \emph{délais} notés $Z^{-1}$ sont présents sur certains arcs dans le réseau afin de retarder d'une étape, la transmission d'une valeur.
Une étape dans un réseau consiste à recevoir une entrée et de générer une sortie.
Il existe plusieurs réseaux de type récurrent (récurrent simple, machine à états liquide, etc).
La Figure \ref{structurermlp} présente un réseau de type perceptron multi-couches récurrent (\rmlp).
Il s'agit d'un \mlp dont les neurones des couches cachées ont un arc qui boucle sur le neurone lui-même avec un $Z^{-1}$ sur cet arc, comme sur la Figure \ref{neuronermlp}.
\subsubsection*{Applications}
Grâce aux délais, le réseau peut approximer des fonctions dont la sortie ne dépend pas seulement de l'entrée actuelle mais aussi des entrées précédentes.
Par exemple, le réseau de neurones récurrent simple (\srn), qui est un \rmlp à une seule couche cachée, peu déja effectuer des prédictions de symbole en suivant une séquence.
Les \emph{machines à états liquides} (\lsm), dont les connections se font de manière aléàtoire, sont utilisés pour la reconnaissance automatique de la parole. %TODO citer
Les \emph{long-short term memory} sont utilisés pour la reconnaissance automatique de la parole ou de l'écriture manuscrite. %TODO citer
%\input{cmac}

\subsection{Le problème du surajustement}
\subsubsection{Le phénomène}
Lorsqu!un réseau de neurones à apprentissage supervisé apprend de trop nombreuses fois à partir du même set d'exemple, alors il peut se produire un phénomène appelé \emph{surajustement}\cite{statistica}.
Le réseau devient dès lors de moins en moins efficace sur des entrées non rencontrées pendant les phases d'apprentissage.
La Figure \ref{interruption} montre qu'au fil des cycles d'apprentissage, l'erreur diminue.
Mais à un certain moment, l'erreur sur des paires entrée/sortie non fournies pendant la phase d'apprentissage augmente. 
C'est à partir de ce moment là que le réseau de neurones surajuste.
\begin{figure}
 \centering
 \includegraphics[scale=0.6]{../figures/surgeneralisation.jpg}
 \caption{Interruption prématurée. \textbf{Source}: STATISTICA Réseaux de Neurones Automatisés (SANN)\cite{statistica}}
 \label{interruption}
\end{figure}
\subsubsection{Les solutions}
\ssstitle{Intérruption prématurée}
Tout d'abord séparons notre ensemble de paires entrée/sortie en deux ensembles.
L'un est nommé \emph{ensemble d'apprentissage} et l'autre \emph{ensemble de test}.
Chacun de ces ensembles doit balayer toutes les valeurs d'entrées possibles.
Ensuite à chaque cycle, on va d'abord effectuer un apprentissage pour chaque paires de l'ensemble d'apprentissage.
Puis on génère des sorties à partir des entrées de l'ensemble de test.
On compare l'erreur de test à celui de l'étape précédente.
L'erreur de test est l'erreur calculée sur les sorties générées et les sorties attendues de l'ensemble de test.
Si l'erreur de test est plus grand que celui du cycle précédent, alors on arrête les cycles d'apprentissage.
Dans la Figure \ref{interruption}, on observe que l'interruption prématurée se produit au moment où son \emph{pouvoir de généralisation}, c'est à dire son efficacité pour des entrées jamais rencontrées, est le plus élevé.\\

Dans la réalité, la courbe d'erreur de test n'est pas aussi lisse que dans la Figure \ref{interruption} mais présente un certain bruit.
Il faut donc pouvoir s'assurer que l'interruption prématurée a bien lieu sur le minimum global de l'erreur de test.
Comparer l'erreur de test à celui du cycle précédent permet de détecter seulement un minimum local.
\ssstitle{Modération des poids}
Cette méthode consiste à pénaliser l'utilisation de trop grand poids.
Pour ce faire, on modifie le calcul d'erreur pour y prendre en compte leur valeur\cite{statistica}.
Soit $E$ l'erreur, soit $W$ un vecteur contenant tous les poids (ne pas prendre les biais en compte), la nouvelle valeur de l'erreur est \[E_{new} = E + \frac{\sigma}{2}W^{T}W\]
où $\sigma$ est une constante appelée \emph{constante de modération}.
Un $\sigma$ trop petit ne permet pas d'éviter le surajustement et un $\sigma$ trop grand empêche la généralisation.


\section{Design}
\subsection{Vecteur d'entrée}

\begin{frame}
 \frametitle{Design du vecteur d'entrée}
 \framesubtitle{Version naïve}
 Pour permettre au réseau de neurones d'approximer les commandes à envoyer, nous devons lui fournir des informations sur l'état actuelle et l'état à atteindre.\\
 \vspace{1cm}
 Un état peut être caractérisé par:
 \begin{itemize}
  \item La position (2 réels)
  \item La vitesse (2 réels)
  \item L'accélération (2 réels)
  \item L'orientation (1 angle en dregré)
 \end{itemize}
 Nous avons pour le moment une entrée de dimension 14.
\end{frame}

\begin{frame}
 \frametitle{Design du vecteur d'entrée}
 \framesubtitle{Relative à la position}
 %les expériences se feront sur un sol plat et partout de même matière
 Nous pouvons remplacer les données \{position actuelle, position à atteindre\} par \{position à atteindre - position actuelle\}.\\
 La position actuelle est donc à retirer.\\
 \vspace{0.5cm}
 \begin{center}
 \begin{tabular}{ll}
  $\rightarrow$ & Entrée de dimension 14,\\
   & Domaine de la position à atteindre réduit.
 \end{tabular}
 \end{center}
\end{frame}

\begin{frame}
 \frametitle{Design du vecteur d'entrée}
 \framesubtitle{Puis relative à l'orientation}
 À chaque instant, pivotons le repère pour que l'axe x soit confondue avec l'axe de l'orientation.\\
 L'orientation actuelle est donc à retirer.\\
 \vspace{0.5cm}
 \begin{center}
 \begin{tabular}{ll}
  $\rightarrow$ & Entrée de dimension 13,\\
   & Domaine de toutes les entrées réduit.
 \end{tabular}
 \end{center}
\end{frame}

\begin{frame}
 \frametitle{Design du vecteur d'entrée}
 \framesubtitle{Possibilité d'une symétrie orthogonale}
 Si on considère que le pattern des données ont une symétrie d'axe x, on inverse toutes les ordonnées des entrées dont la position à atteindre est dans les x négatifs.\\
 \vspace{0.5cm}
 \begin{center}
 \begin{tabular}{ll}
  $\rightarrow$ & Domaine de toutes les entrées réduites de moitié.
 \end{tabular}
 \end{center}
\end{frame}

\begin{frame}
 \frametitle{Choix des entrées}
 Données en entrée lors de l'expérimentation:
 \begin{itemize}
  \item La position à atteindre
  \item La vitesse actuelle
  \item L'accélération actuelle
 \end{itemize}
 \begin{center}
  $\rightarrow$ Entrée de dimension 6.
 \end{center}
\end{frame}

\subsection{Vecteur de sortie}

\begin{frame}
 \frametitle{Design du vecteur de sortie}
 \framesubtitle{Commande \{moteur A, moteur B\}}
 Deux manières de commander la Sphero:
 \begin{enumerate}
  \item En commandant la puissance du moteur A et du moteur B.
  \item En commandant la vitesse et l'orientation.
 \end{enumerate}
 \vspace{0.3cm}
 \begin{center}
 \large Commande \{moteur A, moteur B\} \normalsize\\
 \begin{tabular}{|l|l|l|}
 \hline
 \textbf{Avantages} & \textbf{Inconvénients}\\
 \hline
 \tabitem Plus précis & \tabitem Stabilisateur désactivé\\
 \tabitem Permet plus de dérapage & \tabitem Risque de commande\\
 \tabitem Grande vitesse dans les & \og inutiles \fg\\
 virages & \tabitem Risque une déstabilisation\\
 & totale\\
 \hline
\end{tabular}
\end{center}
\end{frame}

\begin{frame}
 \frametitle{Design du vecteur de sortie}
 \framesubtitle{Commande \{vitesse, orientation\}}
 \begin{center}
 \large Commande \{vitesse, orientation\} \normalsize\\
 \begin{tabular}{|l|l|l|}
 \hline
 \textbf{Avantages} & \textbf{Inconvénients}\\
 \hline
 \tabitem Stabilisateur activé & \tabitem Moins précis\\
 \tabitem Toutes les commandes ont & \tabitem Évite les dérapages\\
 un effet &\\
 \hline
\end{tabular}\\
\vspace{0.5cm}
Choisissons la commande par vitesse et orientation.
\end{center}
\end{frame}

\subsection{Fréquence de streaming}
\newcommand{\inchist}[1]{
 \begin{center}
  \includegraphics[height=6.5cm]{../figures/perf#1.png}
 \end{center}
}
\newcommand{\incbefore}[1]{
 \begin{center}
  \includegraphics[height=6.5cm]{../figures/perf#1Before.png}
 \end{center}
}

\begin{frame}
 \frametitle{Choix de la fréquence de streaming}
 \framesubtitle{Stabilité de la période}
 \inchist{5}
\end{frame}

\begin{frame}
 \frametitle{Choix de la fréquence de streaming}
 \framesubtitle{Pertes de paquet}
 \incbefore{5}
\end{frame}

\section{Choix du réseau}
Nous n'avons pas besoin d'un réseau de neurones récurrent.
Les commandes à appliquer ne dépendent pas de la vitesse, position et autres données fournies aux étapes précédentes.
Le réseau le mieux adapté pour notre problème est un réseau de neurones à base radial.
En effet, les \rbf sont moins sensibles au bruit que les \mlp \cite{adversarial,Gauthier}.%gauthier p 39,40
Car dans la couche cachée, chaque neurone envoit une sortie indiquant la proximité entre le vecteur d'entrée et son prototype.
Donc typiquement, la contribution à l'erreur est presque entièrement celle du neurone ayant son prototype le plus proche de l'entrée.
Les autres neurones cachés auront une modification insignifiante de l'écart-type et du prototype.\\

Un des inconvénients des \rbf est que le nombre de neurones cachés croît avec la dimension et la taille du vecteur d'entrée.
Ceci ne pose pas de problème dans notre cas car le nombre de dimensions est faible.

\section{Architecture}
\subsection{Problème du maître distant}

\begin{frame}
 \frametitle{Le problème du maître distant}
 Un \emph{maître} est ce qui permet de fournir une erreur entre la sortie d'un réseau de neurones et la sortie attendue.\\
 \vspace{0.5cm}
 Dans notre cas, il n'est pas possible de comparer directement la sortie générée avec une sortie attendue.
\end{frame}

\begin{frame}
 \frametitle{Légende des différentes architectures}
 \begin{center}
  \includegraphics[height=6cm]{../figures/applegendes.jpg}
 \end{center}
\end{frame}

\begin{frame}
 \frametitle{Reproduction d'un contrôlleur}
 \incinslide{}
\end{frame}

\begin{frame}
 \frametitle{Apprentissage spécialisé}
 
\end{frame}

\begin{frame}
 \frametitle{Apprentissage en deux phases}
 
\end{frame}

\begin{frame}
 \frametitle{Apprentissage indirecte}
 
\end{frame}

\begin{frame}
 \frametitle{Apprentissage par modèle différentiable}
 
\end{frame}

\begin{frame}
 \frametitle{Appentissage sur plusieurs étapes} % Perhaps useless
 
\end{frame}

\section{Choix de la méthode}
Vu toutes les inconnues mécaniques du mouvement de la Sphero, il s'avère difficile d'établir une équation permettant de prédire de manière précise l'état suivant en fonction des commandes exécutées.
Nous préfèrons donc nous passer d'un modèle analytique.
Dans ce cas, nous ne pouvons pas utiliser la méthode de l'apprentissage spécialisé.
De plus, télécommander une Sphero pour suivre une trajectoire est très difficile pour un humain.
Uiliser un apprentissage par reproduciton de contrôleur n'est donc pas envisageable.\\
Nous pouvons obtenir un maître distant à chaque étape et donc nous n'avons pas besoin de l'architecture de Nguyen et Widrow.\\
Le de convergence vers une solution triviale, impliquée dans le cas d'un apprentissage indirect (section \ref{sec:appindirect}) exclu l'utilisation de cette méthode dans le cadre de ce projet.\\

Il nous reste deux méthodes pour lesquelles l'adaptabilité est sacrifiée: l'apprentissage en deux phases et apprentissage utilisant un modèle différentiable.
Les deux sont envisageables. Les tests seront effectués toujours sur le même sol horizontal.\\

La méthode avec modèle différentiable apportera un nouveau facteur influençant la réussite du projet: la fiabilité du modèle.
Mais puisque le sytème ne change pas, lorsqu'on arrive à obtenir un modèle fiable, nous pouvons effectuer un apprentissage en-ligne qui permettra à la Sphero de parfaire le suivi d'une trajectoire si cette trajectoire est ammenée à être répétée.
Nous utiliserons dans un premier temps la méthode en deux phases pour obtenir un ensemble de paires cible/commande où la précision dépend seulement des capteurs et pas de la fiabilité d'un modèle.

\section{Les API}
Les APIs officiels sont:\cite{SDKofficiels} \textbf{Objective-C}, \textbf{Swift}, \textbf{Android}.\\
Il existe également des APIs créés par la communauté:\cite{gosphero} \textbf{C\#}, \textbf{JavaScript}, \textbf{Ruby}, \textbf{Python}\cite{pythonAPI}, \textbf{Arduino}, \textbf{C++}\cite{cppAPI}.\\

L'API choisie est l'API \textbf{C++} pour le langage de programmation connu et performant exécutable sur PC.

\section{Avancement du projet et perspective}
\begin{itemize}
 \item Un modèle de réseau de neurone a été choisi en excluant ceux qui ne sont pas destinés à notre application et en comparant les deux modèles envisageables: \mlp et \rbf.
 \item Deux méthodes visant à éviter le problème de surajustement ont été évoqués.
 \item Deux méthodes de résolutions pour contourner le problême de maître distant ont été retenus.
 \item Un design de vecteur d'entrée a été établi selon un compromis entre précision du mouvement et convergence de la solution.
 \item Un temps d'échantillonage pour notre ensemble d'apprentissage a été déterminé.
 \item Une implémentation d'un \rbf a débuté. Le paradigme orienté-objet est utilisé de sorte à pouvoir ajouter rapidement l'implémentation d'un \mlp (si le besoin se fait sentir) et obtenir un réseau différentiable sans réécriture de code.
\end{itemize}
Les tâches principales restantes sont:
\begin{itemize}
 \item Terminer l'implémentation du \rbf et la tester
 \item Implémentation d'un programmation d'édition de trajectoire
 \item Obtenir l'ensemble d'apprentissage
 \item Trouver les paramètres optimaux pour le \rbf
 \item Déploiement
\end{itemize}


\subsection*{Remerciements}
\noindent Je remercie Monsieur Pierre \textsc{Hauweele} de m'avoir proposé un projet qui est exactement dans le domaine que je voulais, de son aide et de la direction de mon projet.\\

\noindent Je remercie Monsieur Hadrien Mélot pour la direction de mon projet et de l'aide qu'il m'a apporté.\\

\noindent Je remercie Monsieur Tom Mens d'être rapporteur de mon projet et de son aide.\\

\bibliographystyle{unsrt-fr}%unsrt-fr pour reference fr par num. plainnat-fr ou frplain Pour reference fr par auteur et annee
\bibliography{../bibli}
\appendix

\section{Équations de rétropropagation d'un MLP}\label{sec:eqmlp}
La modification à apporter au poids $j$ du neurone $i$ vaut
\begin{equation}\label{eq:Delta}
 \Delta W_{ij} = -\eta\partiel{Q}{W_{ij}}
\end{equation}
\begin{equation}\label{eq:gradient}
 \partiel{Q}{W_{ij}} = \partiel{Q}{\phi_i} \partiel{\phi_i}{v_i} \partiel{v_i}{W_{ij}}
\end{equation}
Et on a posé
\begin{equation}\label{eq:deltai}
 \delta_i = \partiel{Q}{\phi_i} \partiel{\phi_i}{v_i}
\end{equation}
En reprenant \eqref{eq:gradient}, nous avons maintenant que
\begin{equation}\label{eq:graddelta}
 \partiel{Q}{W_{ij}} = \delta_i \partiel{v_i}{W_{ij}}
\end{equation}
Nous allons déterminer la valeur de $\delta_i$ et de $\partiel{v_i}{W_{ij}}$.
Commençons par déterminer $\partiel{v_i}{W_{ij}}$.
Soit $x_{ij}$ la $j$\textsuperscript{ième} entrée du neurone $i$,
\begin{equation}\label{eq:gradsum}
 \begin{split}
  \partiel{v_i}{W_{ij}} & = \partiel{(W_{i0}x_{i0})}{W_{ij}} + \partiel{(W_{i1}x_{i1})}{W_{ij}} + ... + \partiel{(W_{ij}x_{ij})}{W_{ij}} + ... + \partiel{(W_{im}x_{im})}{W_{ij}}\\
  ~ & = x_{ij}
  \end{split}
\end{equation}
Ensuite, déterminons la valeur de $\delta_i$ selon deux cas: si $i$ est un neurone de sortie et si $i$ est un neurone caché.\\

\textbf{Si le neurone $i$ est un neurone de sortie,}
\begin{equation}\label{eq:sortie1}
 \begin{split}
  \partiel{Q}{\phi_i} & = \frac{1}{2} \left( \partiel{(\phi_0 - s_0)^2}{\phi_i} + \partiel{(\phi_1 - s_1)^2}{\phi_i} + ... + \partiel{(\phi_i - s_i)^2}{\phi_i} + ... + \partiel{(\phi_l - s_l)^2}{\phi_i} \right)\\
  ~ & = \frac{1}{2}2(\phi_i - s_i)\partiel{(\phi_i - s_i)}{\phi_i}\\
  ~ & = (\phi_i - s_i)
 \end{split}
\end{equation}
Soit $\phi'$ la dérivée de $\phi$,
\begin{equation}\label{eq:sortie2}
 \partiel{\phi_i}{v_i} = \phi'(v_i)
\end{equation}
Par \eqref{eq:sortie1} et \eqref{eq:sortie2}, on a
\[\delta_i = (\phi_i - s_i)\phi'(v_i)\]\\

\textbf{Si le neurone $i$ est un neurone caché,}
alors soit $n$ le nombre de neurones de la couche suivante (plus proche des neurones de sorties), soit $k$ un neurone de la couche suivante, on sait que $Q$ est fonction composée,
dépendant de $\phi_k$, lui-même dépendant de $v_k$, dépendant lui-même de $\phi_i$.
C'est pour cela qu'en appliquant le théorème de dérivée des fonctions composées,
\begin{equation}\label{eq:cache1beforebefore}
 \partiel{Q}{\phi_i} = \sum_{k=1}^{n} \partiel{Q}{\phi_k} \partiel{\phi_k}{v_k} \partiel{v_k}{\phi_i}
\end{equation}
Si on subsitue \eqref{eq:deltai} dans \eqref{eq:cache1beforebefore},
\begin{equation}\label{eq:cache1before}
 \partiel{Q}{\phi_i} = \sum_{k=1}^{n} \delta_k \partiel{v_k}{\phi_i}
\end{equation}
Or $\phi_i$ est l'entrée du neurone $k$ recevant la sortie du neurone $i$. Posons $W_{ki}$ le poids que $k$ attribue à $\phi_i$.
On peut alors continuer à développer \eqref{eq:cache1before} comme suit:
\begin{equation}\label{eq:cache1}
 \partiel{v_k}{\phi_i} = \partiel{W_{ki}\phi_i}{\phi_i} = W_{ki}
\end{equation}
Pour la valeur de $\partiel{\phi_i}{v_i}$, nous pouvons reprendre \eqref{eq:sortie2}.
Et du coup, \[\delta_i = \phi_i'(v_i) \sum_{k=1}^{n} \delta_k W_{ki}\]

Nous connaissons maitenant la modification à appliquer sur $W_{ij}$. Pour résumé, subsituons $\delta_i$ et \eqref{eq:gradsum} à \eqref{eq:gradient},
\begin{equation}\label{eq:subgradient}
 \partiel{Q}{W_{ij}} = \delta_i x_{ij}
\end{equation}
Qu'on subsitue à \eqref{eq:Delta} et nous avons enfin
\begin{equation}\label{eq:mlpretro}
 \Delta W_{ij} = -\eta \delta_i x_{ij} \text{~où~}\left\{
  \begin{array}{lll}
   \delta_i & = (\phi_i - s_i)\phi'(v_i) & \text{Si~} i \text{~est un neurone de sortie}\\
   \delta_i & = \phi_i'(v_i) \sum_{k=1}^{n} \delta_k W_{ki} & \text{Si~} i \text{~est un neurone caché}
  \end{array}
 \right.
\end{equation}

\section{Équation de rétropropagation d'un RBF}\label{sec:eqrbf}
Les modifications à apporter aux paramètres d'un neurone $i$ vaut
\begin{equation}\label{eq:modifpoid}
 \Delta W_{ij} = -\eta \partiel{Q}{W_{ij}}
\end{equation}
\begin{equation}\label{eq:modifmu}
 \Delta \mu_{ij} = -\eta \partiel{Q}{\mu_{ij}}
\end{equation}
\begin{equation}\label{eq:modifsigma}
 \Delta \sigma_{ij} = \eta \partiel{Q}{\sigma_{ij}}
\end{equation}

D'abord déterminons le facteur $\partiel{Q}{W_{ij}}$ dans \eqref{eq:modifpoid}.
On est dans le cas où $i$ est un neurone de sortie.
Par \eqref{eq:Q}, $Q$ dépend de $\phi_i$.
Par \eqref{eq:sortiephi}, $phi_i$ dépend de $W_{ij}$.
Par le théorème de dérivée des fonctions composés,
\begin{equation}\label{eq:rbfsortie}
 \partiel{Q}{W_{ij}} = \partiel{Q}{\phi_i} \partiel{\phi_i}{W_{ij}}
\end{equation}
Par \eqref{eq:sortie1}, on sait que
\[\partiel{Q}{\phi_i} = (\phi_i - s_i)\]
Ensuite,
\begin{equation}\label{eq:poidgrad}
 \begin{split}
  \partiel{\phi_i}{W_{ij}} & = \partiel{\frac{\sum_{r=1}^{m}W_{ir}x_{j}}{\factnorm}}{W_{ij}}\\
  ~ & = \frac{1}{\factnorm} \partiel{\sum_{r=1}^{m}W_{ir}x_{j}}{W_{ij}}\\
  ~ & = \frac{x_j}{\factnorm}
 \end{split}
\end{equation}
Subsituons \eqref{eq:sortie1} et \eqref{eq:poidgrad} dans \eqref{eq:rbfsortie}.
\begin{equation}\label{eq:sortiegrad}
 \partiel{Q}{W_{ij}} = (\phi_i - s_i) \frac{x_j}{\factnorm}
\end{equation}
Et \eqref{eq:sortiegrad} dans \eqref{eq:modifpoid},
\[\Delta W_{ij} = -\eta (\phi_i - s_i) \frac{x_j}{\factnorm}\]\\

Déterminons maintenant le facteur $\partiel{Q}{\mu_{ik}}$ dans \eqref{eq:modifmu}. On est dans le cas où $i$ est un neurone caché.
$Q$ dépend de $\phi_i$ qui lui-même est fonction de $\mu_{ik}$. Par le théorème de dérivée de fonction composés,
\begin{equation}\label{eq:composemu}
 \partiel{Q}{\mu_{ik}} = \partiel{Q}{\phi_i} \partiel{\phi_i}{\mu_{ik}}
\end{equation}
On va déterminer la valeur de $\partiel{Q}{\phi_i}$ et puis de $\partiel{\phi_i}{\mu_{ik}}$.
Soit $j$ un neurone de sortie. On sait que $Q$ dépend de tous les $\phi_j$ et les $\phi_j$ dépendent de $\phi_i$. On sait donc que
\begin{equation}\label{eq:mufact1}
 \partiel{Q}{\phi_i} = \sum_{j} \partiel{Q}{\phi_j}\partiel{\phi_j}{\phi_i}
\end{equation}
Par \eqref{eq:sortie1}, on sait que
\[\partiel{Q}{\phi_j} = (\phi_j - s_j)\]
Et posons $x_r$ l'entrée de $j$ provenant de $i$, c'est à dire $\phi_i = x_r$.
Posons $n$ la dimension de l'entrée de $j$.
Posons aussi $R = \sum_{k=1}^{n}x_k$.
Pour trouver la valeur de $\partiel{\phi_j}{\phi_i}$, nous aurons besoin de trouver $\partiel{\frac{1}{R}}{x_k}$.
\begin{equation}\label{eq:1sr}
 \begin{split}
 \partiel{\frac{1}{R}}{x_k} & = \frac{-1}{R^2}\partiel{R}{x_k}\\
 ~ & = \frac{-1}{R^2} 1
 \end{split}
\end{equation}
Dès lors, utilisons \eqref{eq:1sr} pour simplifier l'équation suivante;
\begin{equation}\label{eq:phiiphij}
 \begin{split}
 \partiel{\phi_j}{\phi_i} & = \partiel{\phi_j}{x_r}\\
 ~ & = \partiel{\left(\frac{\sum_{k=1}^{n}W_{jk}x_k}{\sum_{k=1}{n}x_k}\right)}{x_r}\\
 ~ & = \partiel{\left(\frac{W_{j1}x_1}{R}\right)}{x_r} + \partiel{\left(\frac{W_{j2}x_2}{R}\right)}{x_r} + ... + \partiel{\left(\frac{W_{jr}x_r}{R}\right)}{x_r} + .. + \partiel{\left(\frac{W_{jn}x_n}{R}\right)}{x_r}\\
 ~ & = (W_{j1}x_1)\frac{-1}{R^2} + (W_{j2}x_2)\frac{-1}{R^2} + ... + \left[(W_{jr}x_r)\partiel{\frac{1}{R}}{x_r}+\partiel{(W_{jr}x_r)}{x_r}\frac{1}{R}\right] + ... + (W_{jn}x_n)\frac{-1}{R^2}\\
 ~ & = \left(\sum_{k=1}^{n}W_{jk}x_k\frac{-1}{R^2}\right)-W_{jr}x_r\frac{-1}{R^2} + \left[(W_{jr}x_r)\frac{-1}{R^2}+W_{jr}\frac{1}{R}\right]\\
 ~ & = \frac{-1}{R^2} \left[\left(\sum_{k=1}^{n}W_{jk}x_k\right)-W_{jr}x_r + (W_{jr}x_r) + W_{jr}(-R)\right]\\
 ~ & = \frac{1}{R^2} \left(W_{jr}R - \sum_{k=1}^{n}W_{jk}x_k\right)
 \end{split}
\end{equation}
On substitue \eqref{eq:sortie1} et \eqref{eq:phiiphij} dans \eqref{eq:mufact1} pour obtenir
\begin{equation}\label{eq:mufacto1}
 \partiel{Q}{\phi_i} = \sum_{j} (\phi_j - s_j) \frac{1}{R^2} \left(W_{jr}R - \sum_{k=1}^{n}W_{jk}x_k\right)
\end{equation}

Si nous calculons $\partiel{\phi_i}{\mu_{ik}}$, nous obtenons
\begin{equation}\label{eq:mufacto2}
 \partiel{\phi_i}{\mu_{ik}} = \phi_i \frac{x_k-\mu_{ik}}{\sigma_{ik}^2}
\end{equation}
Et en substituant \eqref{eq:mufacto1} et \eqref{eq:mufacto2} dans \eqref{eq:composemu},
\[\partiel{Q}{\mu_{ik}} = \left[\sum_{i}(\phi_j - s_j) \frac{1}{R^2} \left(W_{jr}R - \sum_{k=1}^{n}W_{jk}x_k\right)\right] \phi_i\frac{x_k-\mu_{ik}}{\sigma_{ik}^2}\]
Qu'on substitue dans \eqref{eq:modifmu}, on obtient la modification à appliquer sur $\mu_{ik}$:
\[\Delta\mu_{ik} = -\eta \left[\sum_{i}(\phi_j - s_j) \frac{1}{R^2} \left(W_{jr}R - \sum_{k=1}^{n}W_{jk}x_k\right)\right] \phi_i\frac{x_k-\mu_{ik}}{\sigma_{ik}^2}\]

Enfin déterminons le facteur $\partiel{Q}{\sigma_{ik}}$ dans \eqref{eq:modifsigma}. On est dans le cas où $i$ est un neurone caché.
$Q$ dépend de $\phi_i$ qui lui-même est fonction de $\sigma_{ik}$. Par le théorème de dérivée de fonction composés,
\begin{equation}\label{eq:composesigma}
 \partiel{Q}{\sigma_{ik}} = \partiel{Q}{\phi_i} \partiel{\phi_i}{\sigma_{ik}}
\end{equation}
Si nous calculons $\partiel{\phi_i}{\sigma_{ik}}$, nous obtenons
\begin{equation}\label{eq:sigmafacto2}
 \partiel{\phi}{\sigma_{ik}} = \phi_i \frac{(x_k-\mu_{ik})^2}{\sigma_{ik}^3}
\end{equation}
On peut substituer \eqref{eq:mufacto1} et \eqref{eq:sigmafacto2} dans \eqref{eq:composesigma}:
\[\left[\sum_{i}(\phi_j - s_j) \frac{1}{R^2} \left(W_{jr}R - \sum_{k=1}^{n}W_{jk}x_k\right)\right] \phi_i \frac{(x_k-\mu_{ik})^2}{\sigma_{ik}^3}\]
Qu'on substitue dans \eqref{eq:modifsigma}, on obtient la modification à appliquer sur $\sigma_{ik}$:
\[\Delta \sigma_{ik} = -\eta \left[\sum_{i}(\phi_j - s_j) \frac{1}{R^2} \left(W_{jr}R - \sum_{k=1}^{n}W_{jk}x_k\right)\right] \phi_i \frac{(x_k-\mu_{ik})^2}{\sigma_{ik}^3}\]

\section{Diagramme implémentation de réseaux de neurones}\label{uml}
\includegraphics[width=\textwidth]{../../uml/neurondiag.png}
Diagramme de classe de réseaux de neurones.\\Généré par \href{http://plantuml.com/class-diagram}{plantUML}

\end{document}

\documentclass[12pt,a4paper,oneside, titlepage]{article}
\usepackage[left=2cm,right=2cm,top=2cm,bottom=2cm]{geometry}
\usepackage[pdftex]{graphicx}
\usepackage[final]{pdfpages}
\usepackage[frenchb]{babel}
\usepackage{xspace}
\usepackage{lmodern}
\usepackage[T1]{fontenc}
\usepackage[utf8]{inputenc}
\usepackage{cite}
\usepackage{hyperref}
\usepackage{amsmath}
\usepackage{amssymb}
\usepackage{amsthm}
\usepackage{booktabs}% http://ctan.org/pkg/booktabs
\usepackage{textcomp}% symbole degré
\newcommand{\tabitem}{~~\llap{\textbullet}~~}
\begin{document}
\newcommand{\spherotitle}[2]{
\begin{titlepage}
\begin{center}
\includegraphics[scale=1.50]{../UMONS.jpg}\\[0.4cm]
\includegraphics[scale=0.30]{../FS_Logo.jpg}\\[3cm]
{\Large Un réseau de neurones pour la}\\
\includegraphics[scale=0.30]{../sphero.jpg}\\
\rule{8cm}{0.5mm}\\[0.5cm]
{\huge \bfseries #1}\\[0.2cm]
\rule{8cm}{0.5mm}\\[7cm]
% Author and supervisor. Come from http://www.jujens.eu/posts/2013/Oct/20/latex-page-garde/
    \begin{minipage}{0.4\textwidth}
      \begin{flushleft} \large
        Jason \textsc{Bury}\\
        130538\\
        Master 1 en\\Sciences informatiques\\
      \end{flushleft}
    \end{minipage}
    \begin{minipage}{0.4\textwidth}
      \begin{flushright} \large
        \emph{Codirecteurs :}~~\\M. Pierre \textsc{Hauweele}\\M. Hadrien \textsc{Mélot}\\
        \emph{Rapporteur : }~~\\M. Tom \textsc{Mens}
      \end{flushright}
    \end{minipage}
   \vfill
  {\large #2}
\end{center}
\end{titlepage}
}

\newcommand{\terminologie}{
\begin{figure}
 \begin{center}
 \underline{\hypertarget{terminologie}{Terminologie}}\\
 \begin{tabular}{|l|l|}
  \hline
  Terme & Nom anglais complet\\
  \hline
  ANN & Artificial Neural Network\\
  ELM & Extreme Learning Machine\\
  FNN & Feedforward Neural Networks\\
  MLP & Multi-Layer Perceptron\\
  RBF & Radial Basis Function\\
  CNN & Convolutional neural Network\\
  RNN & Recurrent Neural Network\\
  SRN & Simple Recurrent Network\\
  ESN & Echo State Network\\
  LSM & Liquid State Machine\\
  SNN & Spiking Neural Network\\
  LSTM & Long Short-Term Memory\\
  BRNN & Bi-directional Recurrent Neural Network\\
  RMLP & Recurrent Multilayer Perceptron\\
  \hline
 \end{tabular}
 \end{center}
 \caption{Terminologie des réseaux de neurones dans la littérature anglophone}
 \label{terminologie}
\end{figure}
}

\newcommand{\termi}[1]{\hyperlink{terminologie}{\uppercase{#1}} }
\newcommand{\rbf}{\termi{rbf}}
\newcommand{\mlp}{\termi{mlp}}

\newcommand{\rna}{\hyperlink{rna}{RNA} }
\newcommand{\ubf}[1]{\textbf{\underline{#1}}}
\newcommand{\enum}[1]{``#1''}%{\og\textbf{#1}\fg}
\newcommand{\captionsource}[2]{
  \caption[{#1}]{
    #1
    \\\hspace{\linewidth}
    \textbf{Source:} #2
  }
}

\spherotitle{Rapport}{Année 2016-2017}
\tableofcontents
\newpage
\section{Introduction}
\subsection{Énoncé}
Ce projet consiste à implémenter un \emph{réseau de neurones artificiels} \hypertarget{rna}{(RNA)} de commander la Sphero.
Il devra effecter n'importe quel trajectoire et aussi des trajectoire à grande vitesse et gèrer les dérapages.
La Sphero sera commandé sur un sol plat sans obstacle.
\subsection{Avantages d'un réseau de neurones}
\begin{itemize}
 \item \textbf{parallélisme}: L'apprentissage et la génération de vecteur de sortie est massivement parallélisable dans un RNA.\cite{corelet,Haykin}
 \item \textbf{Généralisation}: Répond de manière raisonnable à une entrée non rencontrée durant la phase d'apprentissage.\cite{statistica,Haykin}
 \item \textbf{Approximation non-linéaire}: Les RNA présentés dans ce rapport sont des approximateurs universels de fonctions non-linéaires.\cite{Haykin}
 \item \textbf{Adaptabilité}: Un RNA avec apprentissage on-line s'adapte aux changements dans le système.\cite{Haykin}
 \item \textbf{Boite noire}: Un RNA agit comme une boite noire. L'utilisateur n'a pas besoin de connaitre le fonctionnement du réseaux.
 \item \textbf{Tolérance au bruit}: Le bruitage dans la phase d'apprentissage impacte peu les performances d'un RNA.\cite{Haykin}
\end{itemize}
C'est grâce à ces avantages qu'un réseaux de neurones artificiels peut être intéressant pour commander la Sphero.
En effet, Il est très difficile de générer les commandes de manières analytique fidèle à la réalité à cause des trop nombreux paramètres à gèrer (Centre de gravité changeant, frottement, dérapages, équilibre, défauts,...).
Grâce au fait qu'un RNA agit comme une boite noire et est approximateur universel de fonction, nous pouvons nous passer de la conception d'un simulateur ou d'une tentative de formule analytique pour génerer les commandes.
De plus, si l'apprentissage se fait on-line, le réseau de neurones pourra s'adapter si le coefficient de frottement du sol change ou si il y une modification à la Sphero.
Et enfin, nous aurons invévitablement des bruitages dans les données fournies par les capteurs.

\newpage
\section{Théorie}
\subsection{Modèles de réseaux de neurones artificiels}
\terminologie
Tout d'abord, analysons les différents modèles de réseau de neurones artificiels éxistants afin de pouvoir sélectionner celui qui convient le mieux à notre application.
Les plus connus d'entre eux sont le perceptron multi-couche et la fonction à base radiale, tous deux non récurrents.
Ensuite nous verrons les réseaux de neurones récurrents qui approximent des fonctions qui peuvent dépendre de toutes les entrées précédentes.
Les terminologies utilisées proviennent de la littérature anglophone et tous ceux rencontrés sont répertoriés à la table \ref{terminologie}.
\newcommand{\ssstitle}[1]{\begin{center}\large\underline{#1}\normalsize\end{center}}
\subsection{Perceptron mutli-couches}
Dans la suite, nous désignerons un \emph{perceptron multi-couches} par sa terminologie anglaise: \mlp pour \enum{MultiLayer Perceptron}.
\subsubsection{Neurone}
La Figure \ref{neuronemlp} schématise le travail d'un neurone d'indice $k$.
\begin{figure}
 \centering
 \includegraphics[scale=0.5]{../figures/neurone.jpg}
 \caption{Un neurone artificiel. \textbf{Source}: Haykin\cite{Haykin}}
 \label{neuronemlp}
\end{figure}
Un neurone effectue tout d'abord une somme pondérée de ses entrées \[v_k = b_k+\sum_{j=1}^{m}x_{j}w_{kj}\] où $x$ est le vecteur d'entrée de dimensions $m$.\\
Chaque terme $x_j$ est mutiplié par un poids $w_{kj}$.
Ce sont les poids qui seront modifiés lors de la phase d'apprentissage.
Le biais $b_k$ est souvent ajouté à la pondération.
Mais pour simplifier les formules, nous pouvons considèrer $b_k$ comme étant l'entrée $x_0 = 1$ de poid fixe $w_{k0} = 1$.
Et la somme pondérée est donc maintenant de la forme \[v_k = \sum_{j=0}^{m}x_{j}w_{kj}\]
Ensuite le neurone applique la \emph{fonction d'activation} $\phi$ sur la somme pondérée.
Le domaine de $y$ est généralement $[0,1]$ ou $[-1,1]$.\cite{Haykin,statistica}
La fonction $\phi$ utilisée dépend du problème qu'on veut résoudre (Table \ref{mlpfonc}).
Par exemple, la fonction Softmax est utilisée en classification.\cite{statistica}%TODO autre fonction

\begin{table}
 \centering
 % tableau de statistica
 \textbf{Fonctions d'activations.} (fonction de $x$)\\
 \begin{tabular}{|l|c|c|}
  Nom & Formule & Image\\
  \hline
  Identité & $x$ & $]-\infty,\infty[$\\
  \hline
  Sigmoïde & $\frac{1}{1+\exp^{-x}}$ & $]0,1]$\\
  \hline
  Tangente hyperbolique & $\frac{\exp^{x}-\exp^{-x}}{\exp^{x}+\exp^{-x}}$ & $]-1,1[$\\
  \hline
  Exponentielle & $e^{-x}$ & $]0,\infty[$\\
  \hline
  Sinus & $\sin{x}$ & $[-1,1]$\\
  \hline
  Softmax & $\frac{\exp^{x}}{\sum{\exp^{x_i}}}$ & $]0,1[$\\
  \hline
 \end{tabular}
 \caption{Fonctions d'activation principalement utilisés dans un MLP. \textbf{Source}: STATISTICA Réseaux de Neurones Automatisés (SANN)\cite{statistica}}
 \label{mlpfonc}
\end{table}
\subsubsection{Structure}
\begin{figure}
 \centering
 \includegraphics[scale=0.5]{../figures/nnstruct.png}
 \caption{Structure MLP à une couche cachée. \textbf{Source}: McCormick\cite{RBFtuto}}
 \label{structuremlp}
\end{figure}
Un \mlp est composé de plusieurs couches (Figure \ref{structuremlp}):
\begin{center}
 couche d'entrée $\rightarrow$ couches cachées $\rightarrow$ couche de sortie.
\end{center}
Un neurone envoie sa sortie vers tous les neurones de la couche suivante.
Il y a autant de neurones d'entrées que la dimension du vecteur d'entrée.
Le $k$\textsuperscript{ième} neurone d'entrée renvoie juste le $k$\textsuperscript{ième} élément du vecteur d'entrée.
Chaque neurone de sortie correspond à une dimension du vecteur de sortie.
Les neurones cachés et neurones de sortie correspondent à la Figure \ref{neuronemlp}.
\subsubsection{Apprentissage supervisé}\label{sec:appmlp}
Il existe plusieurs algorithmes pour changer les poids sur un réseau. Mais le plus connu est l'algorithme de de rétropropagation.\cite{Haykin,Gauthier}
Il s'agit d'un algorithme d'\emph{apprentissage supervisé}.
\begin{definition}
L'apprentissage supervisé est une méthode visant à améliorer un approximateur grâce à un calcul d'erreur (appelé aussi mesure de performance\cite{Gauthier}) comparant une sortie générée avec la sortie attendue.
\end{definition}
Nous avons donc besoin d'un ensemble de paires d'entrées/sorties.
L'algorithme applique la formule développée ci-dessous à tous les neurones sauf les neurones d'entrée.\\

Soit $\phi_i$, la fonction d'activation du neurone $i$. L'erreur entre la sortie produite et la sortie attendue est donnée par $Q$, l'\emph{erreur quadratique} utilisé comme mesure de performance.
\begin{equation} \label{eq:Q}
 Q = \frac{1}{2}\sum_{i}(\phi_i-s_i)^2
\end{equation}
où $i$ est l'indice d'un neurone de sortie et $s_i$ la sortie attendue pour le neurone $i$.

Nous voulons modifier la valeur des poids pour que la prochaine fois que nous prenons la même entrée, l'erreur $Q$ soit plus petite.
Pour chaque poids $W_{ij}$, nous allons calculer le gradient $\partiel{Q}{W_{ij}}$ qui, par définition, indique la façon dont $Q$ varie si $W_{ij}$ augmente d'une valeur $\delta W_{ij}$ infiniment petite.
Ensuite nous \enum{ferons un pas} dans le sens opposé et proportionel au gradient en ajoutant $-\eta\partiel{Q}{W_{ij}}$ à $W_{ij}$ où $\eta$ est une constante appelé \emph{pas du gradient}.
Donc la modification du poid $W_{ij}$ que nous voulons appliquer vaut
\begin{equation}%\label{eq:Delta}
 \Delta W_{ij} = -\eta\partiel{Q}{W_{ij}}
\end{equation}
Déterminons à présent la valeur de ce gradient $\partiel{Q}{W_{ij}}$.
Nous savons que $\phi_i$ est une équation à une variable.
Mais en entrée de cette variable est donnée la somme pondérée $v_i$ et $W_{ij}$ est un des poids de $v_i$.
Par le théorème de dérivation des fonctions composées,
\begin{equation}%\label{eq:gradient}
 \partiel{Q}{W_{ij}} = \partiel{Q}{\phi_i} \partiel{\phi_i}{v_i} \partiel{v_i}{W_{ij}}
\end{equation}
\begin{thm}[dérivation des fonctions composées]
Soit $f:A~\rightarrow~B : y~\rightarrow~f(y)$ et $g:B~\rightarrow~C : x~\rightarrow~g(x)$. Alors la dérivée de $f~\circ~g$ en $x$ vaut
\[\partiel{f}{x} = \partiel{f}{g} \partiel{g}{x}\]
\end{thm}
Posons
\begin{equation}%\label{eq:deltai}
 \delta_i = \partiel{Q}{\phi_i} \partiel{\phi_i}{v_i}
\end{equation}
$\delta_i$ est appelé \emph{contribution à l'erreur} du neurone $i$.
Nous verrons plus tard qu'elle sera utile pour la rétropropagation de la couche prédédente.
Le dévellopement de la formule de rétropropagation a été fait en annexe \ref{sec:eqmlp}.
Et voici ce que nous obtenons au final:\\
\begin{equation}\label{eq:mlpretro}
 \Delta W_{ij} = -\eta \delta_i x_i \text{~où~}\left\{
  \begin{array}{lll}
   \delta_i & = (\phi_i - s_i)\phi'(v_i) & \text{Si~} i \text{~est un neurone de sortie}\\
   \delta_i & = \phi_i'(v_i) \sum_{k=1}^{n} \delta_k W_{ki} & \text{Si~} i \text{~est un neurone caché}
  \end{array}
 \right.
\end{equation}

\subsubsection{Apprentissage non supervisé}
Il existe également des algorithmes d'apprentissage non supervisé.
Ces algorithmes ne cherchent pas à minimiser une erreur mais maximisent un score calculé à partir de la sortie du réseau.
\subsubsection{Apprentissage sur plusieurs MLP}
Lorque l'une des dimensions du vecteur d'entrée est discrète et finie, il est conseillé d'utiliser un \mlp différent par valeur différente sur cette dimension.\cite{Gauthier}
Cette technique peut aussi être utilisée si nous pouvons classer tous les états possibles selon le contexte.
Par exemple, pour la Sphero nous pourrions utiliser un \mlp pour le contexte \enum{en train de déraper} et un autre \mlp pour le contexte \enum{ne dérape pas}.
\subsubsection{Applications}
Les \mlp sont aussi utilisés en commande de robot par caméra.\cite{Pomerleau}
Un \mlp peut aussi être envisagé dans notre application.
Nous lui fournissons une entrée composée d'informations sur l'état actuel du monde, l'état cible et ce réseau de neurones devra nous fournir en sortie la commande à appliquer pour atteindre l'état-cible.

\subsubsection{Fonctions à base radiale}
Nous désignerons un réseau de neurones \emph{fonctions à base radiale} par sa terminologie anglaise: \rbf pour \enum{Rasial Basis Function}.
\newcommand{\factnorm}{\sum_{r=1}^{m}x_{r}}
\ssstitle{Neurone}
Un neurone \textbf{caché} d'un \rbf n'effectue pas de somme pondérée de ses entrées.
Il applique directement sa fonction d'activation $\phi$, une gaussienne de dimension $n$, de moyenne $\mu$ (appelé aussi prototype) et d'écart-type $\sigma$.
\[\phi(x) = e^{-\frac{1}{2}\sum_{k=1}^{n}\frac{(x_k-\mu_{k})^2}{\sigma_{k}^{2}}} \]%\label{eq:cachephi}
%$\phi(x)$ peut se résumer en: \[\phi(x) = e^{-\beta||x-\mu||^2}\]Où $\beta$ est donc un coefficient qui règle la largeur de la courbe en cloche.\\

Un neurone \textbf{de sortie} d'un \rbf n'effectue pas non plus de somme pondérée de ses entrées. Sa fonction d'activation est:
\[\phi(x) = \frac{\sum_{j=1}^{m}W_{j}x_{j}}{\sum_{j=1}^{m}x_{j}}\]%\label{eq:sortiephi}
où $m$ est le nombre de dimension de $x$.
\begin{figure}
 \centering
 \includegraphics[scale=0.7]{../figures/RBFactivation.png}%TODO generer ça soit-même
 \caption{Activation d'un neurone RBF avec différentes valeurs d'écart-type. $\beta=\frac{1}{2\sigma^2}$ \textbf{Source}: McCormick\cite{RBFtuto}}
 \label{rbfactivation}
\end{figure}
La Figure \ref{rbfactivation} représente la sortie d'un neurone \rbf où $\mu$ et $x$ sont de dimension 1 et $\mu = 0$.\\
Pour résumer, un neurone \rbf renvoie une valeur indiquant la similarité entre l'entrée et son prototype.
\ssstitle{Structure}
\begin{figure}
 \centering
 \includegraphics[scale=0.5]{../figures/RBFstruct.png}
 \caption{Structure RBF. \textbf{Source}: McCormick\cite{RBFtuto}}
 \label{structurerbf}
\end{figure}
La structure d'un \rbf est comme celle du \mlp sauf qu'il n'y a qu'une seule couche cachée (Figure \ref{structurerbf}).
\ssstitle{Apprentissage}
Dans ce réseau, les paramètres qui seront modifiés lors de l'apprentissage sont les poids des neurones de sortie et les moyennes et écart-types des neurones cachés.
L'algorithme de rétropropagation peut être utilisé pour un apprentissage supervisé d'un \rbf.
Reprenons \eqref{eq:Q}, l'erreur quadratique $Q$ défini dans la section \ref{sec:appmlp}.
En reprenant le même raisonnement que la rétropropagation dans un \mlp,
on va faire un pas de $\eta$ dans le sens opposé et proportionnel au gradient pour chaque poids $W_{j}$ des neurones de sortie mais aussi des prototypes $\mu_j$ et écart-types $\sigma_j$ des neurones cachés.
C'est à dire que les paramètres seront modifiés de la sorte pour le neurone $i$:
\[\Delta W_{ij} = -\eta \partiel{Q}{W_{ij}}\]
\[\Delta \mu_{ij} = -\eta \partiel{Q}{\mu_{ij}}\]
\[\Delta \sigma_{ij} = -\eta \partiel{Q}{\sigma_{ij}}\]

Le dévellopement des formules de rétropropagation a été fait en annexe \ref{sec:eqrbf}.
Et voici ce que nous obtenons au final:\\
Pour un neurone de sortie :
\[\Delta W_{ij} = -\eta (\phi_i - s_i) \frac{x_j}{\factnorm}\]
Pour un neurone caché, posons $x_r$ l'entrée de $j$ provenant de $i$ (c'est à dire $\phi_i = x_r$),\\
posons $n$ la dimension de l'entrée de $j$,\\
posons aussi $R = \sum_{k=1}^{n}x_k$.
\[\Delta\mu_{ik} = -\eta \left[\sum_{i}(\phi_j - s_j) \frac{1}{R^2} \left(W_{jr}R - \sum_{k=1}^{n}W_{jk}x_k\right)\right] \phi_i\frac{x_k-\mu_{ik}}{\sigma_{ik}^2}\]
\[\Delta \sigma_{ik} = -\eta \left[\sum_{i}(\phi_j - s_j) \frac{1}{R^2} \left(W_{jr}R - \sum_{k=1}^{n}W_{jk}x_k\right)\right] \phi_i \frac{(x_k-\mu_{ik})^2}{\sigma_{ik}^3}\]

\ssstitle{Applications}
Un \rbf peut aussi servir pour la commande de robot\cite{Gauthier}.
Un des inconvénients des \rbf est que le nombre de neurones cachés croît avec la dimension et la taille du vecteur d'entrée puisqu'un neurone caché s'active seulement pour des entrées dans le voisinage de son prototype.
Nous pouvons raisonablement envisager ce modèle pour notre application puisque la description de l'état actuel et de l'état cible contiendra typiquement une coordonnée, un vecteur vitesse, une orientation et éventuellement l'accéleration, etc...
La taille de l'entrée sera donc relativement petite. Ce sera lors de la phase pratique qu'on évaluera le nombre de neurones cachés nécessaires.
De plus, les \rbf sont moins sensibles au bruit\cite{adversarial}.
%\input{cnn}
\subsection{Récurrent (RNN)}
\subsubsection*{Structure}
\begin{figure}
 \centering
 \includegraphics[scale=0.5]{../figures/structurermlp.jpg}
 \caption{Structure RLMP. \textbf{Source}: Haykin. p795\cite{Haykin}}
 \label{structurermlp}
\end{figure}
\begin{figure}
 \centering
 \includegraphics[scale=0.5]{../figures/neuronermlp.jpg}
 \caption{Neurone caché d'un RMLP}
 \label{neuronermlp}
\end{figure}
Un réseau de neurones récurrent est un réseaux présentant des boucles dans sa structure.
Des \emph{délais} notés $Z^{-1}$ sont présents sur certains arcs dans le réseau afin de retarder d'une étape, la transmission d'une valeur.
Une étape dans un réseau consiste à recevoir une entrée et de générer une sortie.
Il existe plusieurs réseaux de type récurrent (récurrent simple, machine à états liquide, etc).
La Figure \ref{structurermlp} présente un réseau de type perceptron multi-couches récurrent (\rmlp).
Il s'agit d'un \mlp dont les neurones des couches cachées ont un arc qui boucle sur le neurone lui-même avec un $Z^{-1}$ sur cet arc, comme sur la Figure \ref{neuronermlp}.
\subsubsection*{Applications}
Grâce aux délais, le réseau peut approximer des fonctions dont la sortie ne dépend pas seulement de l'entrée actuelle mais aussi des entrées précédentes.
Par exemple, le réseau de neurones récurrent simple (\srn), qui est un \rmlp à une seule couche cachée, peu déja effectuer des prédictions de symbole en suivant une séquence.
Les \emph{machines à états liquides} (\lsm), dont les connections se font de manière aléàtoire, sont utilisés pour la reconnaissance automatique de la parole. %TODO citer
Les \emph{long-short term memory} sont utilisés pour la reconnaissance automatique de la parole ou de l'écriture manuscrite. %TODO citer
%\input{cmac}

\subsection{Le problème du surajustement}
\subsubsection{Le phénomène}
Lorsqu!un réseau de neurones à apprentissage supervisé apprend de trop nombreuses fois à partir du même set d'exemple, alors il peut se produire un phénomène appelé \emph{surajustement}\cite{statistica}.
Le réseau devient dès lors de moins en moins efficace sur des entrées non rencontrées pendant les phases d'apprentissage.
La Figure \ref{interruption} montre qu'au fil des cycles d'apprentissage, l'erreur diminue.
Mais à un certain moment, l'erreur sur des paires entrée/sortie non fournies pendant la phase d'apprentissage augmente. 
C'est à partir de ce moment là que le réseau de neurones surajuste.
\begin{figure}
 \centering
 \includegraphics[scale=0.6]{../figures/surgeneralisation.jpg}
 \caption{Interruption prématurée. \textbf{Source}: STATISTICA Réseaux de Neurones Automatisés (SANN)\cite{statistica}}
 \label{interruption}
\end{figure}
\subsubsection{Les solutions}
\ssstitle{Intérruption prématurée}
Tout d'abord séparons notre ensemble de paires entrée/sortie en deux ensembles.
L'un est nommé \emph{ensemble d'apprentissage} et l'autre \emph{ensemble de test}.
Chacun de ces ensembles doit balayer toutes les valeurs d'entrées possibles.
Ensuite à chaque cycle, on va d'abord effectuer un apprentissage pour chaque paires de l'ensemble d'apprentissage.
Puis on génère des sorties à partir des entrées de l'ensemble de test.
On compare l'erreur de test à celui de l'étape précédente.
L'erreur de test est l'erreur calculée sur les sorties générées et les sorties attendues de l'ensemble de test.
Si l'erreur de test est plus grand que celui du cycle précédent, alors on arrête les cycles d'apprentissage.
Dans la Figure \ref{interruption}, on observe que l'interruption prématurée se produit au moment où son \emph{pouvoir de généralisation}, c'est à dire son efficacité pour des entrées jamais rencontrées, est le plus élevé.\\

Dans la réalité, la courbe d'erreur de test n'est pas aussi lisse que dans la Figure \ref{interruption} mais présente un certain bruit.
Il faut donc pouvoir s'assurer que l'interruption prématurée a bien lieu sur le minimum global de l'erreur de test.
Comparer l'erreur de test à celui du cycle précédent permet de détecter seulement un minimum local.
\ssstitle{Modération des poids}
Cette méthode consiste à pénaliser l'utilisation de trop grand poids.
Pour ce faire, on modifie le calcul d'erreur pour y prendre en compte leur valeur\cite{statistica}.
Soit $E$ l'erreur, soit $W$ un vecteur contenant tous les poids (ne pas prendre les biais en compte), la nouvelle valeur de l'erreur est \[E_{new} = E + \frac{\sigma}{2}W^{T}W\]
où $\sigma$ est une constante appelée \emph{constante de modération}.
Un $\sigma$ trop petit ne permet pas d'éviter le surajustement et un $\sigma$ trop grand empêche la généralisation.


\section{Architecture}
\subsection{Problème du maître distant}

\begin{frame}
 \frametitle{Le problème du maître distant}
 Un \emph{maître} est ce qui permet de fournir une erreur entre la sortie d'un réseau de neurones et la sortie attendue.\\
 \vspace{0.5cm}
 Dans notre cas, il n'est pas possible de comparer directement la sortie générée avec une sortie attendue.
\end{frame}

\begin{frame}
 \frametitle{Légende des différentes architectures}
 \begin{center}
  \includegraphics[height=6cm]{../figures/applegendes.jpg}
 \end{center}
\end{frame}

\begin{frame}
 \frametitle{Reproduction d'un contrôlleur}
 \incinslide{}
\end{frame}

\begin{frame}
 \frametitle{Apprentissage spécialisé}
 
\end{frame}

\begin{frame}
 \frametitle{Apprentissage en deux phases}
 
\end{frame}

\begin{frame}
 \frametitle{Apprentissage indirecte}
 
\end{frame}

\begin{frame}
 \frametitle{Apprentissage par modèle différentiable}
 
\end{frame}

\begin{frame}
 \frametitle{Appentissage sur plusieurs étapes} % Perhaps useless
 
\end{frame}

\newpage
\section{Implémentation et application}

\section{Design}
\subsection{Vecteur d'entrée}

\begin{frame}
 \frametitle{Design du vecteur d'entrée}
 \framesubtitle{Version naïve}
 Pour permettre au réseau de neurones d'approximer les commandes à envoyer, nous devons lui fournir des informations sur l'état actuelle et l'état à atteindre.\\
 \vspace{1cm}
 Un état peut être caractérisé par:
 \begin{itemize}
  \item La position (2 réels)
  \item La vitesse (2 réels)
  \item L'accélération (2 réels)
  \item L'orientation (1 angle en dregré)
 \end{itemize}
 Nous avons pour le moment une entrée de dimension 14.
\end{frame}

\begin{frame}
 \frametitle{Design du vecteur d'entrée}
 \framesubtitle{Relative à la position}
 %les expériences se feront sur un sol plat et partout de même matière
 Nous pouvons remplacer les données \{position actuelle, position à atteindre\} par \{position à atteindre - position actuelle\}.\\
 La position actuelle est donc à retirer.\\
 \vspace{0.5cm}
 \begin{center}
 \begin{tabular}{ll}
  $\rightarrow$ & Entrée de dimension 14,\\
   & Domaine de la position à atteindre réduit.
 \end{tabular}
 \end{center}
\end{frame}

\begin{frame}
 \frametitle{Design du vecteur d'entrée}
 \framesubtitle{Puis relative à l'orientation}
 À chaque instant, pivotons le repère pour que l'axe x soit confondue avec l'axe de l'orientation.\\
 L'orientation actuelle est donc à retirer.\\
 \vspace{0.5cm}
 \begin{center}
 \begin{tabular}{ll}
  $\rightarrow$ & Entrée de dimension 13,\\
   & Domaine de toutes les entrées réduit.
 \end{tabular}
 \end{center}
\end{frame}

\begin{frame}
 \frametitle{Design du vecteur d'entrée}
 \framesubtitle{Possibilité d'une symétrie orthogonale}
 Si on considère que le pattern des données ont une symétrie d'axe x, on inverse toutes les ordonnées des entrées dont la position à atteindre est dans les x négatifs.\\
 \vspace{0.5cm}
 \begin{center}
 \begin{tabular}{ll}
  $\rightarrow$ & Domaine de toutes les entrées réduites de moitié.
 \end{tabular}
 \end{center}
\end{frame}

\begin{frame}
 \frametitle{Choix des entrées}
 Données en entrée lors de l'expérimentation:
 \begin{itemize}
  \item La position à atteindre
  \item La vitesse actuelle
  \item L'accélération actuelle
 \end{itemize}
 \begin{center}
  $\rightarrow$ Entrée de dimension 6.
 \end{center}
\end{frame}

\subsection{Vecteur de sortie}

\begin{frame}
 \frametitle{Design du vecteur de sortie}
 \framesubtitle{Commande \{moteur A, moteur B\}}
 Deux manières de commander la Sphero:
 \begin{enumerate}
  \item En commandant la puissance du moteur A et du moteur B.
  \item En commandant la vitesse et l'orientation.
 \end{enumerate}
 \vspace{0.3cm}
 \begin{center}
 \large Commande \{moteur A, moteur B\} \normalsize\\
 \begin{tabular}{|l|l|l|}
 \hline
 \textbf{Avantages} & \textbf{Inconvénients}\\
 \hline
 \tabitem Plus précis & \tabitem Stabilisateur désactivé\\
 \tabitem Permet plus de dérapage & \tabitem Risque de commande\\
 \tabitem Grande vitesse dans les & \og inutiles \fg\\
 virages & \tabitem Risque une déstabilisation\\
 & totale\\
 \hline
\end{tabular}
\end{center}
\end{frame}

\begin{frame}
 \frametitle{Design du vecteur de sortie}
 \framesubtitle{Commande \{vitesse, orientation\}}
 \begin{center}
 \large Commande \{vitesse, orientation\} \normalsize\\
 \begin{tabular}{|l|l|l|}
 \hline
 \textbf{Avantages} & \textbf{Inconvénients}\\
 \hline
 \tabitem Stabilisateur activé & \tabitem Moins précis\\
 \tabitem Toutes les commandes ont & \tabitem Évite les dérapages\\
 un effet &\\
 \hline
\end{tabular}\\
\vspace{0.5cm}
Choisissons la commande par vitesse et orientation.
\end{center}
\end{frame}

\subsection{Fréquence de streaming}
\newcommand{\inchist}[1]{
 \begin{center}
  \includegraphics[height=6.5cm]{../figures/perf#1.png}
 \end{center}
}
\newcommand{\incbefore}[1]{
 \begin{center}
  \includegraphics[height=6.5cm]{../figures/perf#1Before.png}
 \end{center}
}

\begin{frame}
 \frametitle{Choix de la fréquence de streaming}
 \framesubtitle{Stabilité de la période}
 \inchist{5}
\end{frame}

\begin{frame}
 \frametitle{Choix de la fréquence de streaming}
 \framesubtitle{Pertes de paquet}
 \incbefore{5}
\end{frame}

\subsection{Choix du réseau}
%TODO Tableau de comparaison
Nous n'avons pas besoin d'un réseau de neurones récurrent.
Les commandes à appliquer ne dépendent pas de la vitesse, position et autres données fournies aux étapes précédentes.
Le réseau le mieux adapté pour notre problème est un réseau de neurones à base radial.
En effet, les \rbf sont moins sensibles au bruit que les \mlp \cite{adversarial,Gauthier}.%gauthier p 39,40
Car dans la couche cachée, chaque neurone envoit une sortie indiquant la proximité entre le vecteur d'entrée et son prototype.
Donc typiquement, la contribution à l'erreur est presque entièrement celle du neurone ayant son prototype le plus proche de l'entrée.
Les autres neurones cachés auront une modification insignifiante de l'écart-type et du prototype.\\

Un des inconvénients des \rbf est que le nombre de neurones cachés croît avec la dimension et la taille du vecteur d'entrée.
Ceci ne pose pas de problème dans notre cas car le nombre de dimensions est faible.

\subsection{Choix de la méthode}
Vu toutes les inconnues mécaniques du mouvement de la Sphero, il s'avère difficile d'établir une équation permettant de prédire de manière précise l'état suivant en fonction des commandes exécutées.
Nous préfèrons donc nous passer d'un modèle analytique.
Dans ce cas, nous ne pouvons pas utiliser la méthode de l'apprentissage spécialisé.
De plus, télécommander une Sphero pour suivre une trajectoire est très difficile pour un humain.
Utiliser un apprentissage par reproduction de contrôleur n'est donc pas envisageable.\\
Nous pouvons obtenir un maître distant à chaque étape et donc nous n'avons pas besoin de l'architecture de Nguyen et Widrow.\\
Le problème de convergence vers une solution triviale, impliqué dans le cas d'un apprentissage indirect (section \ref{sec:appindirect}) exclu l'utilisation de cette méthode dans le cadre de ce projet.\\

Il nous reste deux méthodes pour lesquelles l'adaptabilité est sacrifiée: l'apprentissage en deux phases et apprentissage utilisant un modèle différentiable.
Les deux sont envisageables. Les tests seront effectués toujours sur le même sol horizontal.\\

La méthode avec modèle différentiable apportera un nouveau facteur influençant la réussite du projet: la fiabilité du modèle.
Mais puisque le sytème ne change pas, lorsqu'on arrive à obtenir un modèle fiable, nous pouvons effectuer un apprentissage en-ligne qui permettra à la Sphero de parfaire le suivi d'une trajectoire si cette trajectoire est ammenée à être répétée.
Nous utiliserons dans un premier temps la méthode en deux phases pour obtenir un ensemble de paires cible/commande où la précision dépend seulement des capteurs et pas de la fiabilité d'un modèle.

\subsection{Les API}
Certaines API permettant l'envoi de commande et réception de message provenant de la Sphero sont disponibles.
Les APIs officiels sont disponibles en: \textbf{Objective-C}, \textbf{Swift}, \textbf{Android}.\cite{SDKofficiels}\\
Il existe également des APIs créés par la communauté disponibles en\cite{gosphero}: \textbf{C\#}, \textbf{JavaScript}, \textbf{Ruby}, \textbf{Python}\cite{pythonAPI}, \textbf{Arduino}, \textbf{C++}\cite{cppAPI}.\\
L'API choisie est l'API \textbf{C++} pour le langage de programmation connu et performant exécutable sur PC.
Malgré les apparences, cet API n'est pas modulaire concernant le streaming.
En effet, le streaming est démarré dès la connection de la Sphero et la lecture des messages de streaming récupérés ne fonctionne que pour le streaming configuré pour l'application que l'auteur voulait en faire.
C'est à dire que des accès à la mémoire se font sans considération des données demandées pour le streaming et ne permet donc que de récupérer la position et la vitesse.
De plus l'écoute des messages de streaming sont non bloquants car ils sont effectués dans un thread à part.
Il n'est donc pas possible, sans modifier l'API, d'utiliser les données reçues dès leur réception.
L'API a donc été modifiée pour récupérer les données qui nous intéressent dans le cadre de ce projet mais aussi pour effectuer le design pattern observer  et le faire hériter d'une interface d'objet capable de streamer.
(Voir la section \ref{sec:impcommander} sur l'implémentation du système de commande.)

Le langage \textbf{Qt} sera utilisé pour créer les interfaces graphiques.
Cela nous permet de créer facilement des interfaces munies de boutons, menus et fenêtres afin de tracer la trajectoire.
De plus, l'implémentation du réseau de neurones pourra être directement utilisée dans un script \textbf{Qt} puisqu'il s'agit d'une extension du c++.

L'API \textbf{Caffe} sera utilisé plus tard afin de modifier facilement l'architecture du réseau de neurones dans le module de commande sans recompilation.\cite{caffe}

\subsection{Implémentation d'un réseau de neurones}
Un \rbf a été implémenté. Bien que fonctionnel, ce n'est pas ce réseau qui est utilisé dans la version actuelle du système de commande de ce projet.
Il a néanmoins été utilisé durant la phase d'expérimentation.
\subsubsection{Propriété utilisée}
Afin d'expliquer la structure de l'implémentation du réseau de neurone, nous allons tout d'abord expliquer pourquoi l'implémentation d'un neurone est indépendant de sa position et du modèle du réseau de neurones.
Et grâce à cette propriété, nous pouvons donc connecter n'importe quelle couche de neurones avec une autre sans modifier l'implémentation.
Nous pourrions donc facilement passer à une méthode d'apprentissage par modèle différentiable (Section \ref{sec:appmodele}) si cela s'avère nécessaire.
De plus, l'implémentation d'un nouveau modèle non récursif comme le \mlp nécésitera juste l'implémentation des nouveaux neurones.
En résumé, l'implémentation d'un réseau \rbf se fera à partir d'une implémentation de réseau de neurones non récursif modulaire.\\

Deux fonctions sont nécessaires pour l'implémentation d'un neurone:
\begin{enumerate}
 \item Une fonction qui permet de générer la sortie à partir de ses entrées (compute).
 \item Une fonction permettant l'apprentissage par rétropropagation (backpropagation).
\end{enumerate}
La fonction qui génère la sortie n'a besoin que des sorties des neurones de la couche précédente.
Et si on observe le développement des formules de rétropropagation dans les annexes \ref{sec:eqmlp} et \ref{sec:eqrbf},
la fonction qui permet l'apprentissage n'a besoin que de la sortie générée précédemment (cela peut être gardé en mémoire) et la \emph{contribution à l'erreur} des neurones de la couche suivante.
Le terme de contribution à l'erreur fait réference au terme $\delta$ dans la formule de modification des poids d'un \mlp (Section \ref{sec:appmlp}).
En effet, pour tous les différents neurones que nous avons vu dans ce rapport, soit $p$ un paramètre à modifier pour l'apprentissage du neurone $i$.
Soit $f_i$ la fonction appliquée aux entrées du neurone $i$ afin de générer la sortie.
$p$ doit être modifié selon la formule suivante:
\[\Delta p_i = -\eta \partiel{Q}{p_i}\]
Et afin de calculer la valeur de $\partiel{Q}{p_i}$, deux facteurs sont à connaître: $\partiel{Q}{f_i}$ et $\partiel{f_i}{p_i}$.
Pour connaître le facteur $\partiel{f_i}{p_i}$, toutes les données dont nous avons besoin se trouvent dans l'implémentation du neurone $i$ si nous y enregistrons sa sortie précédente.
Tandis que pour le facteur $\partiel{Q}{f_i}$,
si le neurone $i$ est un neurone de sortie, alors $\partiel{Q}{f_i}$ est connu, il s'agit de la dérivée partielle de l'erreur entre $f_i$ et sa valeur atendue.
Sinon $$\partiel{Q}{f_i} = \sum_{k \in \text{couche suivate}}\partiel{Q}{f_k}\partiel{f_k}{f_i}$$
Et donc $\partiel{Q}{f_i}$ peut être calculé à partir d'une implémentation dans les neurones de la couche suivante.

\subsubsection{Diagramme de classe}
\begin{figure}
 \centering
 \includegraphics[width=\textwidth]{../../uml/neurondiag.png}
 \caption{Diagramme de classe du réseaux de neurones. \footnotesize Généré par \href{http://plantuml.com/class-diagram}{plantUML}}
 \label{fig:diagclasse}
\end{figure}
Dans le diagramme de classe, Figure \ref{fig:diagclasse}, dans la déclaration de la fonction \texttt{backpropagation()}, \texttt{errorContribution} est le facteur $\partiel{Q}{f_i}$.
Le tableau retourné est stocké dans \texttt{thisContribution}.
Pour chaque entrée $x_j$ du neurone $i$, la $j$\textsuperscript{ième} valeur du tableau retourné est $\partiel{Q}{f_i}\partiel{f_i}{x_j}$.
Ces valeurs sont utilisées pour effectuer la rétropropagation de la couche précédente.

\subsubsection{Limitation actuelle}
Dans la version actuelle de l'implémentation, les biais ne sont pas supportés, le MLP n'est pas encore fonctionnel et des fuites mémoires ont été détectés.
Les fuites mémoires sont dû à l'allocation de mémoire pour transmettre des valeurs d'une couche à l'autre dans \texttt{Net.cpp}.
L'opérateur \texttt{New} effectue une allocation dynamique et nécessite d'être libérée exlplicitement par le programmeur.

\subsection{Implémentation d'un système de commande}\label{sec:impcommander}
Passons maintenant à la description des différentes classes et interfaces du système de commande.

\newcommand{\classname}[1]{\textbf{\texttt{#1}}}
\subsubsection{Structure}
\begin{figure}
 \centering
 \includegraphics[width=0.8\textwidth]{../figures/commander.jpg}
 \caption{Structure du système de commande}
 \label{fig:commander}
\end{figure}
\begin{figure}
 \centering
 \includegraphics[width=\textwidth]{../../uml/commanderdiag.png}
 \caption{Diagramme de classe dy système de commande.\footnotesize Généré par \href{http://plantuml.com/class-diagram}{plantUML}}
 \label{fig:commanderdiag}
\end{figure}

En résumé, on inscrit le commander par machine learning à un Streamer.
Ensuite on démare le streaming en donnant la fréquence $f$ au Streamer.
Le Streamer va périodiquement notifier le commander d'une nouvelle frame de données issues du streaming.
Le commander va demander au Target l'état cible à atteindre dans $\frac{1}{f}$ secondes d'après l'état actuel du Streamer.
Le commander Passe ensuite l'état actuel et l'état à atteindre dans un DataAdapter qui a pour but de résumer les données et les fournir comme input au réseau de neurones.
Ce DataAdapter se charge aussi de réduire le domaine d'entrée pour avoir une convergence plus rapide.
Le commander récupère le output du réseau de neurones et le passe au DataAdapter qui éffectuera une transformation afin d'obtenir les commandes à envoyer directement au Streamer.
Cette procédure est illustré dans la Figure \ref{fig:commander} est reprend les éléments suivants:
\begin{itemize}
 \item \classname{StreamFrame}: Une structure contenant les données brutes du streaming.
  C'est à dire l'orientation de la Sphero, sa position, son vecteur vitesse et son vecteur d'accélération mais aussi le temps entre la récéption de cette frame et la précédente.
 \item \classname{StreamObserver}: Un objet qui pourra être notifié par un Streamer chaque fois que une \texttt{StreamFrame} est reçue.
  LearningCommander mais aussi le commander par commandes aléatoires sont des StreamObservers. (Figure \ref{fig:commanderdiag})
 \item \classname{Streamer}: Un objet capable de fournir des données issues d'un streaming et les envoyer au StreamObserver qui lui est abonné.
  La Sphero et VirtualSphero sont des streamers. (Figure \ref{fig:commanderdiag})
 \item \classname{Target}: Fourni un état à atteindre d'après l'état actuelle de la Sphero.
 \item \classname{LearningCommander}: Il essaie d'atteindre l'état demandé par Target
 \item \classname{DataAdapter}: Transforme les données de streaming et de l'état à atteindre afin de réduire le domaine d'entrée des données à fournir au réseau de neurones.
  Permet la transformation inverse.
 \item \classname{TransformedFrame}: Une structure contenant les données de streaming mais relatif à la position actuelle de la Sphero et relatif à son orientation,
  avec les données nécessaires sur l'état à atteindre.
 \item \classname{Net}: Le réseau de neurones.
\end{itemize}

\subsubsection{Génération de commande aléatoire}
Afin d'obtenir une base de données qui permetra d'entrainer préalablement le réseau de neurones,
On va d'abord la piloter de manière aléatoire et récuperer ses données de streaming ainsi que les commandes fournies pour y arriver.
Le comportement de ce générateur a un impact important sur la qualitée des données.
Ceci a été observé expérimentalement.

\ssstitle{Éviter les colisions}
Tous les générateurs de commandes aléatoire implémentés durant ce projet ont été conçu pour éviter que la Sphero sorte d'une zone rectangulaire afin d'éviter les colisions.
Lorsque la Sphero dépasse sa zone, des commandes sont envoyées afin que la Sphero fasse demi-tour avec une orientation sufisament proche de celle qui le permetra de revenir dans la zone.
Par exemple, si la Sphero dépasse le bord droit de la zone,
alors l'orientation commandée est dans l'intervale $270 \pm \Delta\text{max}_{\text{about-turn}}$ où $\Delta\text{max}_{\text{about-turn}}$ est un paramètre inférieur à 90 degrés.

\ssstitle{Les erreurs influençant la qualité des données}
Tout d'abord il faut éviter que la différence entre l'orientation actuelle de la Sphero et celle qui est commandée soit supérieur à ce que la Sphero est capable de faire avec sa vitesse angulaire maximale.
En effet, supposons que la Sphero soit capable de changer son orientation de maximum 30 degrés sur un temps de $\frac{1}{f}$ où $f$ est la fréquence de streaming.
Alors toutes les commandes dont le changement d'orientation est entre 30 et 180 degrés, à partir d'un même état initial, aboutiront au même état-cible.
C'est à dire que plusieurs commandes de valeur significativement différentes sont optimales pour atteindre cet état-cible.
Ce qui est mauvais pour la convergence du réseau de neurones.
Si plusieurs outputs sont optimales pour le même input, alors la solution du réseau de neurones ne convergera pas vers un de ces outputs mais vers un outputs de valeur entre ces deux outputs optimals.
Cela n'empêchera pas de produire un modèle performant au niveau de la commande puisque à partir du moment où l'orientation commandée est trop élevée pour que la Sphero puisse l'atteindre en $\frac{1}{f}$ secondes, sa valeur n'influence pas la conduite.
Par contre c'est pour le calcul d'erreur que cela pose problème.
Et non seulement cela empêche une bonne convergence mais cela élargi aussi le domaine d'entrée.
Expérimentalement, sans limiter la différence d'orientation, ni le réseau de neurones implémenté, ni le \mlp de Weka, ni celui du package nnet de R ne sont capables de fournir une solution correcte.
Leur comportement est de retourner la moyenne des outputs, quelque soit l'input.\\


Ensuite, afin d'effectuer des trajectoires aléatoires mais naturelles et utiles pour l'apprentissage, il faut pouvoir simuler un humain qui effectuerait des mouvements aléatoires sur une télécommande.
En effet, si la différence d'orientation entre l'état actuelle et l'orientation commandée est purement aléatoire, alors elle alterne rapidement entre positif et négatif.
Ce qui provoque des tremblements, des mouvements balanciers et des trajectoires trop souvent en ligne droite.
Les données de streaming issues d'un tel comportement ne permettaient pas non plus au réseau implémenté et au \mlp de Weka de converger vers une autre solution que la moyenne de tous les outputs.
Pour pouvoir simuler cette commande naturelle, le générateur va d'abord choisir une orientation aléatoire.
Ensuite un nombre d'étapes aléatoires.
Soit $N$ le nombre d'étapes, soit $d$ la différence d'angle (celle inférieur à 180\textdegree) entre l'orientation actuelle de la Sphero et la ciblée.
Le générateur va commander une rotation de $\frac{d}{N}$ pour les $N$ prochaines commandes sauf au moment où la Sphero quitte sa zone délimitée.
Le nombre $N$ est généré de sorte à éviter le premier problème cité.
Ensuite, lorsque la Sphero atteind l'orientation cible, elle a une chance sur trois de changer d'orientation cible,
afin d'éffectuer quand-même des mouvements droits.
La génération de commande sur la vitesse se fait de la même manière.

Et pour finir, il y a une ambiguïté sur le format de l'angle obtenu par streaming.
En effet, dans la documentation, il est écrit que pour commander une nouvelle orientation, celle-ci doit être un angle entre 0\textdegree et 359\textdegree et dont le sens est le suivant:
0\textdegree sigifie droit devant, 90\textdegree à droite, 180\textdegree en arrière et 270\textdegree à gauche.\cite{SDKofficiels}
L'angle obtenu par streaming (et uniquement par streaming) est entre -179\textdegree et 180\textdegree.
Mais il n'y pas d'information concernant le sens.
Il a été conclu que le sens des angles échantillonnés est l'inverse de l'angle à envoyer en commande en observant qu'en commandant l'orientation à 90\textdegree, la Sphero fini par échantillonner -90\textdegree.

\subsubsection{Récupération de la trajectoire}
Passons ensuite au commander utilisant le réseau de neurones.
La Sphero ne sera pratiquemment jamais exactement sur la position cible.
Actuellement, pour résoudre ce problème, l'objet Target cherche parmis les points de sa trajectoire, la plus proche de la position actuelle.
Il renvoie ensuite l'état juste après celui dont la position est le plus proche de la position actuelle de la Sphero.
Mais cette technique présente plusieurs inconvénients:
Elle ne permet pas les trajectoires croisées comme les trajectoires en 8.
De plus, la Sphero pourrait dévier assez fort de sa trajectoire cible.
Dans ce cas, la position de la Sphero risque d'être plus proche d'un autre point de la trajectoire que celui qu'il est sensé atteindre.
%\ssstitle{Méthode}
%\ssstitle{Limitation}
%\ssstitle{Solution à la limitation} TODO
\subsubsection{Éditeur de trajectoire}
\begin{figure}
 \centering
 \includegraphics[scale=0.6]{../figures/targettrajectory.png}
 \caption{Éditeur de trajectoire}
 \label{fig:targettrajectory}
\end{figure}
Une interface graphique a été implémentée en Qt afin de créer des trajectoires à la souris. (Figure \ref{fig:targettrajectory})


\subsection{Expérimentation}
Observons ensuite le résultat de l'apprentissage des réseaux de neurones sur les données issues de la commande aléatoire.

\newcommand{\actu}[1]{#1_{\text{actuelle}}}
\newcommand{\target}[1]{#1_{\text{target}}}
\newcommand{\xactu}{\actu{X}}
\newcommand{\yactu}{\actu{Y}}
\newcommand{\vactu}{\actu{V}}
\newcommand{\vtarget}{\target{V}}
\newcommand{\oactu}{\actu{O}}
\newcommand{\otarget}{\target{O}}
\newcommand{\mpwekawidth}{.48\linewidth}
\newcommand{\incweka}[1]{\includegraphics[scale=0.5]{#1}}
\newcommand{\wecaption}[1]{\caption{#1.\footnotesize (Généré par Weka 3.8.1)\normalsize}}
\subsubsection{Test du réseau de neurones}
Afin de vérifier si le réseau de neurones implémenté ne contient pas d'erreur, il sera d'abord utilisé pour une Sphero virtuelle.
Un modèle très simple de Sphéro a été implémenté. Ce modèle ne simule pas la vitesse et l'inertie angulaire.
Le modèle est le suivant: Soit
\begin{itemize}
 \item $(\xactu, \yactu)$ la position actuelle en cm,
 \item $\vactu$ la vitesse actuelle en cm/s
 \item $\vtarget$ la vitesse commandée d'unité inconnue,
 \item $\oactu$ l'orientation en degrés,
 \item $\otarget$ l'orientation commandée en degrés,
 \item $T$ la période de streaming.
\end{itemize}
\[ \text{acceleration} = a(\vtarget - \vactu)\]
Où $a \in \mathbb{R}^{+}$ est un paramètre.
\[ \text{Nouvelle vitesse} = \vactu + \text{acceleration} \times T \]
Posons $d$ la différence entre $\otarget$ et $\oactu$, négatif si le sens de $\oactu \rightarrow \otarget$ est horlogique.
Alors
\[ \text{Nouvelle orientation} = \oactu + d \]
Nouvelle vitesse et Nouvelle orientation sont limitées selon un paramètre.
\begin{center}
 \begin{tabular}{ll}
  Nouvelle position = & $(\xactu + (\text{Nouvelle vitesse})\cos(\text{Nouvelle orientation})$,\\
   & $\yactu + \text{Nouvelle vitesse}\sin(\text{Nouvelle orientation}))$
 \end{tabular}
\end{center}

Deux réseaux de neurones ont été testés: le \rbf implémenté et le \mlp de Weka.
Le fonction d'activation des neurones cachés de Weka est la sigmoïde.
\[ f(x) = \frac{1}{1+e^{-x}} \]
Tandis que la fonction d'activation des neurones de sortie est une fonction linéaire.

\begin{figure}
 \begin{minipage}[c]{\mpwekawidth}
  \includegraphics[width=\textwidth]{../figures/virtualResultSpeed.png}
 \end{minipage}
 \begin{minipage}[c]{\mpwekawidth}
  \includegraphics[width=\textwidth]{../figures/virtualHeadResult.png}
 \end{minipage}
 \wecaption{Nuages de point prédit/espéré sur la vitesse et l'orientation sur les données de la Sphero virtuelle}
 \label{we:virtualResult}
\end{figure}
Tous les deux étaient capables d'approximer ce modèle.
Pour que les résultats observés ne soient pas biaisées par le surajustement, la technique de 10fold cross-validation a été utilisée.
C'est à dire que la base de données est séparée en 10 ensembles de tailles égales.
Pour chaque ensemble, on construit le modèle sur les 90 autres pourcents de données et on utilise ces 10\% comme ensemble de test.
À la fin, on obtient un nuage de points (sortie espérée, sortie aproximée par le modèle) où nous pouvons, par exemple,
calculer le coefficient de corrélation pour avoir un indice sur la qualité du réseau de neurones pour ces données.
Comme sur la Figure \ref{we:virtualResult} où la corrélation est de 0,99 pour la vitesse et l'orientation.
Deux modèles différents ont été construit car il a été observé que le RBF implémenté est plus éfficace si il n'y a qu'une sortie à générer au lieu de deux.

\subsubsection{Observation des données réelles}
\begin{figure}
 \centering
 \incweka{../figures/targetxyvirtual.png}
 \wecaption{Nuage des positions et de la direction à commander pour les atteindre. Pour la Sphero virtuelle}
 \label{we:targetvirtual}
\end{figure}
\begin{figure}
 \centering
 \incweka{../figures/targetxSpeedvirtual.png}
 \wecaption{Nuage des positions et de la vitesse à commander pour les atteindre. Pour la Sphero virtuelle}
 \label{we:targetvirtualSpeed}
\end{figure}

Observons les données fournies par la commande aléatoire sur la véritable Sphero.
Dans la figure \ref{we:targetxy} et \ref{we:targetxySpeed}, chaque point représentent la position que doit atteindre la Sphero $\frac{1}{f}$ secondes plus tard.
Pour chaque point, la Sphero est en (0,0) et est dirigée vers la droite, parallèle à l'axe x.
La couleur du point représente la valeur de l'orientation à commander.
Plus le ton est orangé, plus la valeur est grande.
On pourrait s'attendre à ce que plus le point est vers le haut, plus la valeur de head est grande, comme on peut l'observer sur les données de commande aléatoire sur la Sphero virtuelle. (Figure \ref{we:targetvirtual})
On pourrait s'attendre aussi à ce que plus le point à atteindre est éloigné de la position actuelle, plus la vitesse commandée doit être élevée, comme on peut également le voir sur les données venant de la Sphero virtuelle. (Figure \ref{we:targetvirtualSpeed})

\begin{figure}
 \centering
 \incweka{../figures/targetxy.png}
 \wecaption{Nuage des positions et de la direction à commander pour les atteindre. Pour la Sphero réelle}
 \label{we:targetxy}
\end{figure}
Concernant les données de la Sphero réelle, si on prend les attributs deux à deux, on n'observe pas de pattern quant à la direction à prendre selon les données acquises.
Une de ces observation est illustrée dans la Figure \ref{we:targetxy}.
Dans cette figure, tous les points ont été légèrement déplacés aléatoirement afin de pouvoir tous les distinguer.
Nous devrions voir en plus de 3 dimensions pour observer un éventuel pattern.
Mais en tout cas, selon Wela, il y en a un.
C'est ce que nous allons voir dans \hyperref[sec:realdataex]{l'expérimentation sur les données réelles}.

\begin{figure}
 \centering
 \incweka{../figures/targetxySpeed.png}
 \wecaption{Nuage des positions et de la vitesse à commander pour les atteindre. Pour la Sphero réelle}
 \label{we:targetxySpeed}
\end{figure}
Tandis que pour la vitesse à commander, sur la Figure \ref{we:targetxySpeed} on observe effectivement une tendance à commander une vitesse plus élevée pour atteindre des positions plus éloignées.
Plus la couleur du point est orangée, plus la vitesse est élevée.

\subsubsection{Expérimentation sur données réelles}\label{sec:realdataex}
Puisque cette expérience s'est déroulée en même temps que la phase de conception et de test du générateur de commande aléatoire,
il a fallut préparer un espace accessible sur une assez longue période.
La vitesse a été bridée et aucune caméra n'a été utilisée.
Dans ces conditions, une fréquence de 40Hz était trop élevée pour capturer les différences de positions car l'odométrie est capturée en nombre entier en cm.
Une fréquence de 5Hz était assez basse pour observer les différences de position mais assez élevée pour effectuer des mouvements fluides.
Pour toutes les obsevations effectuées, le RBF implémenté retournait toujours la moyenne des outputs.

\ssstitle{Avec données relatives à la position et l'orientation}
\begin{minipage}[c]{\mpwekawidth}
 \includegraphics[width=\textwidth]{../figures/speed121314N1500H20.png}
 \begin{center}
  \textbf{Vitesse}
 \end{center}
\end{minipage}
\begin{minipage}[c]{\mpwekawidth}
 \includegraphics[width=\textwidth]{../figures/121314N1500H20.png}
 \begin{center}
  \textbf{Orientation}
 \end{center}
\end{minipage}
\\

Ici, les données d'entrées sont relatifs à la position de la Sphero et à son orientation.
Les données sont normalisées avant des les utiliser dans le réseau de neurones.
Si les valeurs de sortie ne sont pas normalisées, alors si elles ont un domaine différent, elles n'auront pas le même poids lors du calcul d'erreur.
Si les valeurs d'entrée ne sont pas normalisées, alors nous devons paramètrer nous-mêmes l'initialisation de certains neurones.
Comme les neurones du RBF, si ils sont initialisés avec un écart-type de 1 et une moyenne de 0 mais que les valeurs d'entrées peuvent atteindre plusieurs centaines,
alors malgré la précision en 64bits du calcul de la fonction d'activation, elles retourneront souvent 0.
Idem pour la fonction sigmoïde, elle retourna souvent 0 ou 1.

En construisant le modèle en règlant les paramètres, un optimum local a été atteind sur Weka. Il s'agit d'une seule couche de 20 neurones cachés avec un learning rate (i.e. pas de gradient) de 0,3 et un nombre d'époques de 1500.
(C'est à dire que les données d'entrainement sont présentés 1500 fois aux réseaux de neurones).
Le coefficient de corrélation du modèle construit par Weka est de 0.5404 pour la vitesse et de 0,69 pour l'orientation.

\ssstitle{Sans normalisation des données}
\begin{minipage}[c]{\mpwekawidth}
 \includegraphics[width=\textwidth]{../figures/121314SpeedOutputNotNormalized.png}
 \begin{center}
  \textbf{Vitesse}
 \end{center}
\end{minipage}
\begin{minipage}[c]{\mpwekawidth}
 \includegraphics[width=\textwidth]{../figures/HeadNotNormalized.png}
 \begin{center}
  \textbf{Orientation}
 \end{center}
\end{minipage}
\\

Si nous n'avions pas normalisé les données, le coefficient de corrélation du modèle construit par Weka est de 0,0119 pour la vitesse et 0,0264 pour l'orientation.
Nous pouvons d'ailleur ci-dessus observer un comportement étrange sur les prédictions données par le modèle.
La configuration du réseau de neurones est la même que la précédente.

\ssstitle{Ajout de la vitesse commandée en attribut}
\incweka{../figures/121314SpeedN1500H20.png}
\\

Revenons aux données normalisées et ajoutons la vitesse commandée en attribut.
Toujours avec la même configuration du réseau de neurones.
Puisque nous pouvons utiliser un réseau différent pas sortie, il est possible de d'abord prédire la vitesse à commander et d'ensuite ajouter cette donnée pour la prédiction de l'orientation à prendre si celle-ci améliore la précision.
Dans ce cas, le coefficient de corrélation atteind 0,72 pour l'orientation.
Un test est nécessaire afin de confirmer si cette augmentation de 0,03 est significatif.
Par ailleur, si on ajoute l'orientation commandée comme attribut pour le modèle prédisant la vitesse à commander, la corrélation est de 0,4991.

\ssstitle{En considérant la symétrie orthogonale}
Dans la section \ref{sec:choixdesign} nous avons évoquer une possibilité d'une symétrie orthogonale d'axe x sur l'orientation et la vitesse à prendre.
C'est d'ailleur ce qui est observé sur les données de la Sphero virtuelle, Figure \ref{we:targetvirtual} et \ref{we:targetvirtualSpeed}.
Nous allons le vérifier pour les données réelles.
Pour chaque instance, si l'ordonnée de la position à atteindre est négatif, on multiplie par -1 toutes les ordonnées des entrées.
C'est à dire que si targety est <0, alors targety = -targety, currentSpeedy = -currentSpeedy et currentAccely = -currentAccely.
Toujours avec la même configuration de réseau, le coefficient de corrélation devient 0,5123 pour la vitesse et 0,6339 pour l'orientation.

\ssstitle{Sélection des attributs}
TODO

\section{Conclusion}
Malgré que le projet ne soit pas arrivé à son terme, nous avons découvert que pour pouvoir piloter une Sphero avec un réseau de neurone, nous avons besoin de nouveaux attributs ou d'une autre achitecture de réseau de neurones.
En effet, malgré tous les problèmes rencontrés nuisant à la qualité des données récoltées (section \ref{sec:choixdesign} et \ref{sec:choixdesign}) ,
un réseau de neurone devrait être capable d'approximer une fonction même avec des imprécisions, ou, du moins, de converger vers une autres solution que la moyenne de toutes les sorties.
Les autres attributs qui peuvent être ajoutées sont la vitesse à atteindre, l'inclinaison gauche-droite, l'inclinaison arrière-avant et les commandes précédentes.

Le comportement étrange de Weka, section \ref{sec:realdataex}, nous a peut-être mis sur une mauvaise piste.
Pensant que le problème vient du réseau de neurones implémenté alors qu'il peut provenir des données.
Il faudrait donc à l'avenir éviter d'utiliser ce logiciel pour des tâches de régression.

Dans l'état actuel de ce projet, nous avons un système de commande commandant aléatoirement mais en effectuant des trajectoires naturelles.
Nous pouvons désormais éditer facilement l'architecture du réseau de neurones pour le système de commandes,
éditer des trajectoires,
ou encore integrer de nouvelles transformations de données ou de nouveaux modèles de Sphero virtuelles.

Enfin, le développement d'un réseau de neurones nous a permis de comprendre en détails son fonctionnement et une propriété qui les rend très modulaires.

\subsection*{Remerciements}
\noindent Je remercie Monsieur Pierre \textsc{Hauweele} de m'avoir proposé un projet qui est exactement dans le domaine que je voulais, de son aide et de la direction de mon projet.\\

\noindent Je remercie Monsieur Hadrien Mélot pour la direction de mon projet et de l'aide qu'il m'a apporté.\\

\noindent Je remercie Monsieur Tom Mens pour son feed-back et ses conseils sur la rédaction de ce rapport.\\

\noindent Je remercie Monsieur Gauvain Devillez pour m'avoir débloqué plusieurs fois dans la compilation et ses conseils.sec:realdataex

\bibliographystyle{unsrt-fr}%unsrt-fr pour reference fr par num. plainnat-fr ou frplain Pour reference fr par auteur et annee
\bibliography{../bibli}
\appendix

\section{Équations de rétropropagation d'un MLP}\label{sec:eqmlp}
La modification à apporter au poids $j$ du neurone $i$ vaut
\begin{equation}\label{eq:Delta}
 \Delta W_{ij} = -\eta\partiel{Q}{W_{ij}}
\end{equation}
\begin{equation}\label{eq:gradient}
 \partiel{Q}{W_{ij}} = \partiel{Q}{\phi_i} \partiel{\phi_i}{v_i} \partiel{v_i}{W_{ij}}
\end{equation}
Et on a posé
\begin{equation}\label{eq:deltai}
 \delta_i = \partiel{Q}{\phi_i} \partiel{\phi_i}{v_i}
\end{equation}
En reprenant \eqref{eq:gradient}, nous avons maintenant que
\begin{equation}\label{eq:graddelta}
 \partiel{Q}{W_{ij}} = \delta_i \partiel{v_i}{W_{ij}}
\end{equation}
Nous allons déterminer la valeur de $\delta_i$ et de $\partiel{v_i}{W_{ij}}$.
Commençons par déterminer $\partiel{v_i}{W_{ij}}$.
Soit $x_{ij}$ la $j$\textsuperscript{ième} entrée du neurone $i$,
\begin{equation}\label{eq:gradsum}
 \begin{split}
  \partiel{v_i}{W_{ij}} & = \partiel{(W_{i0}x_{i0})}{W_{ij}} + \partiel{(W_{i1}x_{i1})}{W_{ij}} + ... + \partiel{(W_{ij}x_{ij})}{W_{ij}} + ... + \partiel{(W_{im}x_{im})}{W_{ij}}\\
  ~ & = x_{ij}
  \end{split}
\end{equation}
Ensuite, déterminons la valeur de $\delta_i$ selon deux cas: si $i$ est un neurone de sortie et si $i$ est un neurone caché.\\

\textbf{Si le neurone $i$ est un neurone de sortie,}
\begin{equation}\label{eq:sortie1}
 \begin{split}
  \partiel{Q}{\phi_i} & = \frac{1}{2} \left( \partiel{(\phi_0 - s_0)^2}{\phi_i} + \partiel{(\phi_1 - s_1)^2}{\phi_i} + ... + \partiel{(\phi_i - s_i)^2}{\phi_i} + ... + \partiel{(\phi_l - s_l)^2}{\phi_i} \right)\\
  ~ & = \frac{1}{2}2(\phi_i - s_i)\partiel{(\phi_i - s_i)}{\phi_i}\\
  ~ & = (\phi_i - s_i)
 \end{split}
\end{equation}
Soit $\phi'$ la dérivée de $\phi$,
\begin{equation}\label{eq:sortie2}
 \partiel{\phi_i}{v_i} = \phi'(v_i)
\end{equation}
Par \eqref{eq:sortie1} et \eqref{eq:sortie2}, on a
\[\delta_i = (\phi_i - s_i)\phi'(v_i)\]\\

\textbf{Si le neurone $i$ est un neurone caché,}
alors soit $n$ le nombre de neurones de la couche suivante (plus proche des neurones de sorties), soit $k$ un neurone de la couche suivante, on sait que $Q$ est fonction composée,
dépendant de $\phi_k$, lui-même dépendant de $v_k$, dépendant lui-même de $\phi_i$.
C'est pour cela qu'en appliquant le théorème de dérivée des fonctions composées,
\begin{equation}\label{eq:cache1beforebefore}
 \partiel{Q}{\phi_i} = \sum_{k=1}^{n} \partiel{Q}{\phi_k} \partiel{\phi_k}{v_k} \partiel{v_k}{\phi_i}
\end{equation}
Si on subsitue \eqref{eq:deltai} dans \eqref{eq:cache1beforebefore},
\begin{equation}\label{eq:cache1before}
 \partiel{Q}{\phi_i} = \sum_{k=1}^{n} \delta_k \partiel{v_k}{\phi_i}
\end{equation}
Or $\phi_i$ est l'entrée du neurone $k$ recevant la sortie du neurone $i$. Posons $W_{ki}$ le poids que $k$ attribue à $\phi_i$.
On peut alors continuer à développer \eqref{eq:cache1before} comme suit:
\begin{equation}\label{eq:cache1}
 \partiel{v_k}{\phi_i} = \partiel{W_{ki}\phi_i}{\phi_i} = W_{ki}
\end{equation}
Pour la valeur de $\partiel{\phi_i}{v_i}$, nous pouvons reprendre \eqref{eq:sortie2}.
Et du coup, \[\delta_i = \phi_i'(v_i) \sum_{k=1}^{n} \delta_k W_{ki}\]

Nous connaissons maitenant la modification à appliquer sur $W_{ij}$. Pour résumé, subsituons $\delta_i$ et \eqref{eq:gradsum} à \eqref{eq:gradient},
\begin{equation}\label{eq:subgradient}
 \partiel{Q}{W_{ij}} = \delta_i x_{ij}
\end{equation}
Qu'on subsitue à \eqref{eq:Delta} et nous avons enfin
\begin{equation}\label{eq:mlpretro}
 \Delta W_{ij} = -\eta \delta_i x_{ij} \text{~où~}\left\{
  \begin{array}{lll}
   \delta_i & = (\phi_i - s_i)\phi'(v_i) & \text{Si~} i \text{~est un neurone de sortie}\\
   \delta_i & = \phi_i'(v_i) \sum_{k=1}^{n} \delta_k W_{ki} & \text{Si~} i \text{~est un neurone caché}
  \end{array}
 \right.
\end{equation}

\section{Équation de rétropropagation d'un RBF}\label{sec:eqrbf}
Les modifications à apporter aux paramètres d'un neurone $i$ vaut
\begin{equation}\label{eq:modifpoid}
 \Delta W_{ij} = -\eta \partiel{Q}{W_{ij}}
\end{equation}
\begin{equation}\label{eq:modifmu}
 \Delta \mu_{ij} = -\eta \partiel{Q}{\mu_{ij}}
\end{equation}
\begin{equation}\label{eq:modifsigma}
 \Delta \sigma_{ij} = \eta \partiel{Q}{\sigma_{ij}}
\end{equation}

D'abord déterminons le facteur $\partiel{Q}{W_{ij}}$ dans \eqref{eq:modifpoid}.
On est dans le cas où $i$ est un neurone de sortie.
Par \eqref{eq:Q}, $Q$ dépend de $\phi_i$.
Par \eqref{eq:sortiephi}, $phi_i$ dépend de $W_{ij}$.
Par le théorème de dérivée des fonctions composés,
\begin{equation}\label{eq:rbfsortie}
 \partiel{Q}{W_{ij}} = \partiel{Q}{\phi_i} \partiel{\phi_i}{W_{ij}}
\end{equation}
Par \eqref{eq:sortie1}, on sait que
\[\partiel{Q}{\phi_i} = (\phi_i - s_i)\]
Ensuite,
\begin{equation}\label{eq:poidgrad}
 \begin{split}
  \partiel{\phi_i}{W_{ij}} & = \partiel{\frac{\sum_{r=1}^{m}W_{ir}x_{j}}{\factnorm}}{W_{ij}}\\
  ~ & = \frac{1}{\factnorm} \partiel{\sum_{r=1}^{m}W_{ir}x_{j}}{W_{ij}}\\
  ~ & = \frac{x_j}{\factnorm}
 \end{split}
\end{equation}
Subsituons \eqref{eq:sortie1} et \eqref{eq:poidgrad} dans \eqref{eq:rbfsortie}.
\begin{equation}\label{eq:sortiegrad}
 \partiel{Q}{W_{ij}} = (\phi_i - s_i) \frac{x_j}{\factnorm}
\end{equation}
Et \eqref{eq:sortiegrad} dans \eqref{eq:modifpoid},
\[\Delta W_{ij} = -\eta (\phi_i - s_i) \frac{x_j}{\factnorm}\]\\

Déterminons maintenant le facteur $\partiel{Q}{\mu_{ik}}$ dans \eqref{eq:modifmu}. On est dans le cas où $i$ est un neurone caché.
$Q$ dépend de $\phi_i$ qui lui-même est fonction de $\mu_{ik}$. Par le théorème de dérivée de fonction composés,
\begin{equation}\label{eq:composemu}
 \partiel{Q}{\mu_{ik}} = \partiel{Q}{\phi_i} \partiel{\phi_i}{\mu_{ik}}
\end{equation}
On va déterminer la valeur de $\partiel{Q}{\phi_i}$ et puis de $\partiel{\phi_i}{\mu_{ik}}$.
Soit $j$ un neurone de sortie. On sait que $Q$ dépend de tous les $\phi_j$ et les $\phi_j$ dépendent de $\phi_i$. On sait donc que
\begin{equation}\label{eq:mufact1}
 \partiel{Q}{\phi_i} = \sum_{j} \partiel{Q}{\phi_j}\partiel{\phi_j}{\phi_i}
\end{equation}
Par \eqref{eq:sortie1}, on sait que
\[\partiel{Q}{\phi_j} = (\phi_j - s_j)\]
Et posons $x_r$ l'entrée de $j$ provenant de $i$, c'est à dire $\phi_i = x_r$.
Posons $n$ la dimension de l'entrée de $j$.
Posons aussi $R = \sum_{k=1}^{n}x_k$.
Pour trouver la valeur de $\partiel{\phi_j}{\phi_i}$, nous aurons besoin de trouver $\partiel{\frac{1}{R}}{x_k}$.
\begin{equation}\label{eq:1sr}
 \begin{split}
 \partiel{\frac{1}{R}}{x_k} & = \frac{-1}{R^2}\partiel{R}{x_k}\\
 ~ & = \frac{-1}{R^2} 1
 \end{split}
\end{equation}
Dès lors, utilisons \eqref{eq:1sr} pour simplifier l'équation suivante;
\begin{equation}\label{eq:phiiphij}
 \begin{split}
 \partiel{\phi_j}{\phi_i} & = \partiel{\phi_j}{x_r}\\
 ~ & = \partiel{\left(\frac{\sum_{k=1}^{n}W_{jk}x_k}{\sum_{k=1}{n}x_k}\right)}{x_r}\\
 ~ & = \partiel{\left(\frac{W_{j1}x_1}{R}\right)}{x_r} + \partiel{\left(\frac{W_{j2}x_2}{R}\right)}{x_r} + ... + \partiel{\left(\frac{W_{jr}x_r}{R}\right)}{x_r} + .. + \partiel{\left(\frac{W_{jn}x_n}{R}\right)}{x_r}\\
 ~ & = (W_{j1}x_1)\frac{-1}{R^2} + (W_{j2}x_2)\frac{-1}{R^2} + ... + \left[(W_{jr}x_r)\partiel{\frac{1}{R}}{x_r}+\partiel{(W_{jr}x_r)}{x_r}\frac{1}{R}\right] + ... + (W_{jn}x_n)\frac{-1}{R^2}\\
 ~ & = \left(\sum_{k=1}^{n}W_{jk}x_k\frac{-1}{R^2}\right)-W_{jr}x_r\frac{-1}{R^2} + \left[(W_{jr}x_r)\frac{-1}{R^2}+W_{jr}\frac{1}{R}\right]\\
 ~ & = \frac{-1}{R^2} \left[\left(\sum_{k=1}^{n}W_{jk}x_k\right)-W_{jr}x_r + (W_{jr}x_r) + W_{jr}(-R)\right]\\
 ~ & = \frac{1}{R^2} \left(W_{jr}R - \sum_{k=1}^{n}W_{jk}x_k\right)
 \end{split}
\end{equation}
On substitue \eqref{eq:sortie1} et \eqref{eq:phiiphij} dans \eqref{eq:mufact1} pour obtenir
\begin{equation}\label{eq:mufacto1}
 \partiel{Q}{\phi_i} = \sum_{j} (\phi_j - s_j) \frac{1}{R^2} \left(W_{jr}R - \sum_{k=1}^{n}W_{jk}x_k\right)
\end{equation}

Si nous calculons $\partiel{\phi_i}{\mu_{ik}}$, nous obtenons
\begin{equation}\label{eq:mufacto2}
 \partiel{\phi_i}{\mu_{ik}} = \phi_i \frac{x_k-\mu_{ik}}{\sigma_{ik}^2}
\end{equation}
Et en substituant \eqref{eq:mufacto1} et \eqref{eq:mufacto2} dans \eqref{eq:composemu},
\[\partiel{Q}{\mu_{ik}} = \left[\sum_{i}(\phi_j - s_j) \frac{1}{R^2} \left(W_{jr}R - \sum_{k=1}^{n}W_{jk}x_k\right)\right] \phi_i\frac{x_k-\mu_{ik}}{\sigma_{ik}^2}\]
Qu'on substitue dans \eqref{eq:modifmu}, on obtient la modification à appliquer sur $\mu_{ik}$:
\[\Delta\mu_{ik} = -\eta \left[\sum_{i}(\phi_j - s_j) \frac{1}{R^2} \left(W_{jr}R - \sum_{k=1}^{n}W_{jk}x_k\right)\right] \phi_i\frac{x_k-\mu_{ik}}{\sigma_{ik}^2}\]

Enfin déterminons le facteur $\partiel{Q}{\sigma_{ik}}$ dans \eqref{eq:modifsigma}. On est dans le cas où $i$ est un neurone caché.
$Q$ dépend de $\phi_i$ qui lui-même est fonction de $\sigma_{ik}$. Par le théorème de dérivée de fonction composés,
\begin{equation}\label{eq:composesigma}
 \partiel{Q}{\sigma_{ik}} = \partiel{Q}{\phi_i} \partiel{\phi_i}{\sigma_{ik}}
\end{equation}
Si nous calculons $\partiel{\phi_i}{\sigma_{ik}}$, nous obtenons
\begin{equation}\label{eq:sigmafacto2}
 \partiel{\phi}{\sigma_{ik}} = \phi_i \frac{(x_k-\mu_{ik})^2}{\sigma_{ik}^3}
\end{equation}
On peut substituer \eqref{eq:mufacto1} et \eqref{eq:sigmafacto2} dans \eqref{eq:composesigma}:
\[\left[\sum_{i}(\phi_j - s_j) \frac{1}{R^2} \left(W_{jr}R - \sum_{k=1}^{n}W_{jk}x_k\right)\right] \phi_i \frac{(x_k-\mu_{ik})^2}{\sigma_{ik}^3}\]
Qu'on substitue dans \eqref{eq:modifsigma}, on obtient la modification à appliquer sur $\sigma_{ik}$:
\[\Delta \sigma_{ik} = -\eta \left[\sum_{i}(\phi_j - s_j) \frac{1}{R^2} \left(W_{jr}R - \sum_{k=1}^{n}W_{jk}x_k\right)\right] \phi_i \frac{(x_k-\mu_{ik})^2}{\sigma_{ik}^3}\]

\end{document}

\subsection{Expérimentation}
Observons ensuite le résultat de l'apprentissage des réseaux de neurones sur les données issues de la commande aléatoire.

\newcommand{\actu}[1]{#1_{\text{actuelle}}}
\newcommand{\target}[1]{#1_{\text{target}}}
\newcommand{\xactu}{\actu{X}}
\newcommand{\yactu}{\actu{Y}}
\newcommand{\vactu}{\actu{V}}
\newcommand{\vtarget}{\target{V}}
\newcommand{\oactu}{\actu{O}}
\newcommand{\otarget}{\target{O}}
\subsubsection{Test du réseau de neurones}
Afin de vérifier si le réseau de neurones implémenté ne contient pas d'erreur, il sera d'abord utilisé pour une Sphero virtuelle.
Un modèle très simple de Sphéro a été implémenté. Ce modèle ne simule pas la vitesse et l'inertie angulaire.
Le modèle est le suivant: Soit
\begin{itemize}
 \item $(\xactu, \yactu)$ la position actuelle en cm,
 \item $\vactu$ la vitesse actuelle en cm/s
 \item $\vtarget$ la vitesse commandée d'unité inconnue,
 \item $\oactu$ l'orientation en degrés,
 \item $\otarget$ l'orientation commandée en degrés,
 \item $T$ la période de streaming.
\end{itemize}
\[ \text{acceleration} = a(\vtarget - \vactu)\]
Où $a \in \mathbb{R}^{+}$ est un paramètre.
\[ \text{Nouvelle vitesse} = \vactu + \text{acceleration} T \]
Posons $d$ la différence entre $\otarget$ et $\oactu$, négatif si le sens de $\oactu \rightarrow \otarget$ est horlogique.
Alors
\[ \text{Nouvelle orientation} = \oactu + d \]
Nouvelle vitesse et Nouvelle orientation sont limitée selon un paramètre.
\begin{center}
 \begin{tabular}{ll}
  Nouvelle position = & $(\xactu + (\text{Nouvelle vitesse})\cos(\text{Nouvelle orientation})$,\\
   & $\yactu + \text{Nouvelle vitesse}\sin(\text{Nouvelle orientation}))$
 \end{tabular}
\end{center}

\subsubsection{Observation des données réelles}
Observons les données fournies par la commande aléatoire sur la véritable Sphero.
Dans la figure ? chaque point représente la position que doit atteindre la Sphero 200ms plus tard.
Pour chaque point, la Sphero est en (0,0) et est dirigée vers la droite, parallèle à l'axe x.
On pourrait s'attendre à ce que plus le point est vers le haut, plus la valeur de head est grande, comme on peut l'observer sur les données de commande aléatoire sur la Sphero virtuelle TODO.
Mais ce n'est pas le cas, on observe un nuage de points aléatoires.
Pour chaque attribut qu'on peut prendre 2 à 2, on n'observe pas de pattern.
Il nous faudrait pouvoir voir en plus que 3 dimensions pour, peut-être observer un pattern.
Mais en tout cas, il y en a un.
C'est ce que nous allons voir dans l'expérimentation sur les données réelles.

\subsubsection{Expérimentation sur données réelles}
TODO\\
\ssstitle{Avec données relatives à la position et l'orientation}
\ssstitle{Avec données relatives à la position et l'orientation et ordered speed en attribut}
\ssstitle{Avec données relatives à la position et l'orientation, ordered speed en attribut et en mirroir sur l'axe x}
\ssstitle{Avec données relatives à la position et l'orientation, ordered speed en attribut et output non normalisé}
\ssstitle{Avec données relatives à la position et l'orientation, ordered speed en attribut et input non normalisé}
\ssstitle{Avec données relatives à la position et l'orientation, ordered speed en attribut, input et output non normalisés}
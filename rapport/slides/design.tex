\section{Design}
\subsection{Vecteur d'entrée}

\begin{frame}
 \frametitle{Design du vecteur d'entrée}
 \framesubtitle{Version naïve}
 Pour permettre au réseau de neurones d'approximer les commandes à envoyer, nous devons lui fournir des informations sur l'état actuelle et l'état à atteindre.\\
 \vspace{1cm}
 Un état peut être caractérisé par:
 \begin{itemize}
  \item La position (2 réels)
  \item La vitesse (2 réels)
  \item L'accélération (2 réels)
  \item L'orientation (1 angle en dregré)
 \end{itemize}
 Nous avons pour le moment une entrée de dimension 14.
\end{frame}

\begin{frame}
 \frametitle{Design du vecteur d'entrée}
 \framesubtitle{Relative à la position}
 %les expériences se feront sur un sol plat et partout de même matière
 Nous pouvons remplacer les données \{position actuelle, position à atteindre\} par \{position à atteindre - position actuelle\}.\\
 La position actuelle est donc à retirer.\\
 \vspace{0.5cm}
 \begin{center}
 \begin{tabular}{ll}
  $\rightarrow$ & Entrée de dimension 14,\\
   & Domaine de la position à atteindre réduit.
 \end{tabular}
 \end{center}
\end{frame}

\begin{frame}
 \frametitle{Design du vecteur d'entrée}
 \framesubtitle{Puis relative à l'orientation}
 À chaque instant, pivotons le repère pour que l'axe x soit confondue avec l'axe de l'orientation.\\
 L'orientation actuelle est donc à retirer.\\
 \vspace{0.5cm}
 \begin{center}
 \begin{tabular}{ll}
  $\rightarrow$ & Entrée de dimension 13,\\
   & Domaine de toutes les entrées réduit.
 \end{tabular}
 \end{center}
\end{frame}

\begin{frame}
 \frametitle{Design du vecteur d'entrée}
 \framesubtitle{Possibilité d'une symétrie orthogonale}
 Si on considère que le pattern des données ont une symétrie d'axe x, on inverse toutes les ordonnées des entrées dont la position à atteindre est dans les x négatifs.\\
 \vspace{0.5cm}
 \begin{center}
 \begin{tabular}{ll}
  $\rightarrow$ & Domaine de toutes les entrées réduites de moitié.
 \end{tabular}
 \end{center}
\end{frame}

\begin{frame}
 \frametitle{Choix des entrées}
 Données en entrée lors de l'expérimentation:
 \begin{itemize}
  \item La position à atteindre
  \item La vitesse actuelle
  \item L'accélération actuelle
 \end{itemize}
 \begin{center}
  $\rightarrow$ Entrée de dimension 6.
 \end{center}
\end{frame}

\subsection{Vecteur de sortie}

\begin{frame}
 \frametitle{Design du vecteur de sortie}
 \framesubtitle{Commande \{moteur A, moteur B\}}
 Deux manières de commander la Sphero:
 \begin{enumerate}
  \item En commandant la puissance du moteur A et du moteur B.
  \item En commandant la vitesse et l'orientation.
 \end{enumerate}
 \vspace{0.3cm}
 \begin{center}
 \large Commande \{moteur A, moteur B\} \normalsize\\
 \begin{tabular}{|l|l|l|}
 \hline
 \textbf{Avantages} & \textbf{Inconvénients}\\
 \hline
 \tabitem Plus précis & \tabitem Stabilisateur désactivé\\
 \tabitem Permet plus de dérapage & \tabitem Risque de commande\\
 \tabitem Grande vitesse dans les & \og inutiles \fg\\
 virages & \tabitem Risque une déstabilisation\\
 & totale\\
 \hline
\end{tabular}
\end{center}
\end{frame}

\begin{frame}
 \frametitle{Design du vecteur de sortie}
 \framesubtitle{Commande \{vitesse, orientation\}}
 \begin{center}
 \large Commande \{vitesse, orientation\} \normalsize\\
 \begin{tabular}{|l|l|l|}
 \hline
 \textbf{Avantages} & \textbf{Inconvénients}\\
 \hline
 \tabitem Stabilisateur activé & \tabitem Moins précis\\
 \tabitem Toutes les commandes ont & \tabitem Évite les dérapages\\
 un effet &\\
 \hline
\end{tabular}\\
\vspace{0.5cm}
Choisissons la commande par vitesse et orientation.
\end{center}
\end{frame}

\subsection{Fréquence de streaming}
\newcommand{\inchist}[1]{
 \begin{center}
  \includegraphics[height=6.5cm]{../figures/perf#1.png}
 \end{center}
}
\newcommand{\incbefore}[1]{
 \begin{center}
  \includegraphics[height=6.5cm]{../figures/perf#1Before.png}
 \end{center}
}

\begin{frame}
 \frametitle{Choix de la fréquence de streaming}
 \framesubtitle{Stabilité de la période}
 \inchist{5}
\end{frame}

\begin{frame}
 \frametitle{Choix de la fréquence de streaming}
 \framesubtitle{Pertes de paquet}
 \incbefore{5}
\end{frame}